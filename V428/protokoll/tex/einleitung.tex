\section{Einleitung}\label{sec:einleitung}
\pagenumbering{arabic}
In diesem Versuch wird sich mit Röntgenstrahlung und ihrer Anwendung in der Materialanalyse und der Untersuchung von kristallinen Strukturen auseinandergesetzt.\par
Im ersten Versuchsteil werden die Wellenlängen und Energien der charakteristischen Röntgenstrahlung einer unbekannten Röntgenröhre mithilfe der Bragg-Reflexion
an einem NaCl-Einkristall bestimmt. So ist es möglich, das Material der Anode der verwendeten Röntgenröhre zu ermitteln. Außerdem wird die Feinstruktur
der $\mathrm{K}_{\alpha}$-Linie von Molybdän in der vierten Beugungsordnung untersucht, wobei der Wellenlängenabstand innerhalb des Dubletts bestimmt wird.\par
Im zweiten Versuchsteil werden mit einem Röntgenenergiedetektor die Fluoreszenzspektren von vier unbekannten Legierungen und ebenso einige Referenzspektren
aufgenommen. Mithilfe der Referenzspektren können dann die einzelnen Komponenten der vier Legierungen bestimmt werden. Außerdem werden hier
die Massenanteile der einzelnen Komponenten einer der unbekannten Legierungen bestimmt.\par
Im dritten Versuchsteil wird mithilfe eines Röntgenfilms eine Laue-Aufnahme eines NaCl-Kristalls durchgeführt, um dessen Symmetrie und Gitterstruktur zu untersuchen.
\section{Zerstörungsfreie Analyse chemischer Zusammensetzungen}\label{sec:materialanalyse}
In diesem Versuchsteil werden mit einem Röntgenenergiedetektor die Fluoreszenzspektren von vier verschiedenen unbekannten Legierungen
und einiger Referenzspektren aufgenommen. Damit lassen sich die einzelnen Komponenten der vier unbekannten Legierungen bestimmen.
Außerdem werden die Massenanteile der einzelnen Komponenten einer dieser unbekannten Legierungen bestimmt.\par
Das aufgenommene Fluoreszenzspektrum einer Legierung entsteht dadurch, dass die mit der Röntgenröhre erzeugte Röntgenstrahlung
Elektronen mit hohen Bindungsenergien aus den Atomen der Legierung herausschlagen kann, sodass diese frei gewordenen Energieniveaus mit
Elektronen aus höheren Schalen wieder aufgefüllt werden können. Durch diese atomaren Übergange wird dann wieder Röntgenstrahlung
emittiert, was in dem Fluoreszenzspektrum resultiert. Da dieses Fluoreszenzspektrum Informationen über die atomaren Übergänge
der Legierungs-Atome beinhält, können so Rückschlüsse auf Materialeigenschaften geschlossen werden.
\subsection{Aufbau}\label{subsec:material_aufbau}
Für die Materialanalyse wird die Cu-Röntgenröhre verwendet. Außerdem wird für die Messung der Röntgenenergiedetektor benutzt, welcher anstelle
des Sensorarms mit dem Zählrohr eingebaut wird.\par
Bei einem Röntgenenergiedetektor handelt es sich im Wesentlichen um eine PIN-Photodiode.
Eine PIN-Photodiode ist eine PN-Diode (Zusammensetzung einer p-dotierten und einer n-dotieren Halbleiterschicht), wobei zwischen
p- und n-dotierter Schicht eine intrinsische Halbleiterschicht (kaum bis nur gering dotiert) eingesetzt wird. Die Röntgenphotonen
erzeugen in der intrinsischen Halbleiterschicht Elektron-Loch-Paare, wobei die Elektronen zu der n-dotierten Schicht und die Löcher
zu der p-dotierten Schicht abgezogen werden. Je energiereicher die Röntgenphotonen sind, desto mehr weitere Elektron-Loch-Paare können durch
Stoßprozesse, ausgelöst durch das ursprünglich entstandene Elektron-Loch-Paar, erzeugt werden. Somit ist der gemessene Strom
proportional zur Energie der einfallenden Röntgenstrahlung. Ein weiterer Bestandteil des Röntgenenergiedetektors ist ein
Vielkanalanalysator, welcher die unterschiedlichen gemessenen Energien einem Kanal zuordnet, sodass schließlich ein Energiespektrum zu
betrachten ist.\par
In die Kollimatoraufnahme wird ein Kollimator mit \SI{1}{\milli \meter} Spaltbreite eingesetzt.
Dann werden die Abstände zwischen Spaltblende des Kollimators und Drehachse sowie zwischen Drehachse und Eintrittsöffnung des Energiedetektors jeweils
auf ca. 5-6 \unit{\cm} eingestellt. Schließlich wird am Röntgengerät der Taster \textit{COUPLED} gedrückt und der Winkel mit dem Dreheinsteller
\textit{Adjust} auf \SI{45}{\degree} eingestellt, sodass das Target bei \SI{45}{\degree} und der Sensor bei \SI{90}{\degree} steht.
Der hier verwendete Versuchsaufbau ist in \cref{fig:aufbau_material} dargestellt.
\begin{figure}[H]
	\centering
	\includegraphics[width=0.6\linewidth]{../figs/aufbau_material.png}
	\caption{Versuchsaufbau zur Materianalyse mit einem Röntgenenergiedetektor:
    a - Kollimator, b - Target, c - Targettisch, d - Röntgenenergiedetektor \cite{material_handblatt}.}
	\label{fig:aufbau_material}
\end{figure}
\subsection{Messung}\label{subsec:material_messung}
Zuerst wird das Kalibriertarget (FeZn-Plättchen) auf den Targettisch gelegt. Zur Aufnahme der Messwerte wird am PC das Programm \textit{Cassy Lab}
verwendet. In \textit{Cassy Lab} wird die Vielkanalmessung mit 512 Kanälen und negativen Pulsen aktiviert. Außerdem wird eine Verstärkung von
\num{-2,5} und eine Messdauer von \SI{180}{\second} eingestellt. Am Röntgengerät wird $U = \SI{35,0}{\kilo \volt}$ und $I = \SI{1,00}{\milli \ampere}$
eingestellt. Dann wird mit dem Schalter HV ON/OFF die Hochspannung eingeschaltet. Dann kann die Spektrumaufnahme am PC gestartet werden.\par
Mit den selben Messparametern werden die Spektren der vier unbekannten Legierungen und die Referenzspektren der Elemente Blei (Pb), Eisen (Fe),
Gold (Au), Indium (In), Kupfer (Cu), Nickel (Ni), Silber (Ag), Titan (Ti), Wolfram (W), Zinn (Su), Zink (Zn) und Zirkonium (Zr) aufgenommen.
\subsection{Auswertung}\label{subsec:material_auswertung}
\subsubsection*{Energiekalibrierung}\label{subsubsec:energie_kali}
Zunächst werden analog zu der in \cref{sec:bragg} verwendeten Methode an alle aufgenommenen Fluoreszenzspektren (Kalibrations-Spektrum der FeZn-Legierung, 13 Referenzspektren, Spektren vier unbekannter Legierungen)
Gauß-Kurven angepasst. Alle Referenzspektren inklusive Gauß-Kurven-Anpassung sind im \hyperref[sec:anhang]{Anhang} aufgeführt. Eine detaillierte Diskussion
jeder einzelnen Gauß-Kurven-Anpassung führt hier zu weit, allerdings ist in den Grafiken zu den Röntgenfluoreszenzspektren aller Legierungen visuell zu erkennen (bzw. siehe auch $\chi^2$), dass die Gauß-Kurven-Anpassungen
die Messwerte sehr gut beschreiben.\par
Nun wird das Spektrum des FeZn-Plättchens zur Energieeichung verwendet, da in diesem Spektrum die $\mathrm{K}_{\alpha}$- sowie $\mathrm{K}_{\beta}$-Linien
von Eisen und Zink deutlich zu sehen sind. Das Fluoreszenzspektrum der FeZn-Legierung inklusive Gauß-Kurven-Anpassung ist in \cref{fig:fezn} dargestellt.
Die zugehörigen Anpassungsparameter sind in \cref{tab:fezn-gauss-fits} zu finden.
\begin{figure}[H]
	\centering
	\includegraphics[width=0.6\linewidth]{../figs/FeZn.pdf}
	\caption{Röntgenfluoreszenzspektrum der FeZn-Legierung.}
	\label{fig:fezn}
\end{figure}
\begin{table}[H]
    \centering
 \caption{Gauß-Anpassungen an das FeZn-Spektrum.}
 \begin{tabular}{c c c c c}
 \hline Linie & Digitaler Kanal $K$ (Mittelwert) & Standardabweichung $\sigma$ & Höhe der Gauß-Kurve $H$ & $\chi^2$ \\ 
 \hline
 1 & $\num{83.57\pm 0.09}$ & $\num{2.52\pm 0.08}$ & $\num{10700\pm 600}$ & $\num{7.45}$ \\
 2 & $\num{88.1\pm 0.2}$ & $\num{5.0\pm 0.1}$ & $\num{4300\pm 300}$ & $\num{7.45}$ \\
 3 & $\num{112.62\pm 0.13}$ & $\num{2.7\pm 0.1}$ & $\num{1430\pm 80}$ & $\num{3.78}$ \\
 4 & $\num{125.7\pm 1.3}$ & $\num{3.9\pm 1.8}$ & $\num{200\pm 300}$ & $\num{1.11}$ \\
 \hline\end{tabular}
 \label{tab:fezn-gauss-fits}
\end{table} In Abbildung \cref{fig:fezn} ist zu erkennen, dass Linie 1 und Linie 2 ineinander liegen, weshalb an diese beiden Linien eine Überlagerung
zweier Gauß-Kurven angepasst wird. Für diese Anpassung ist die Anpassungsgüte von $\chi^2 = \num{7.45}$ im Vergleich zu den Anpassungsgüten der anderen
beiden Anpassungen am schlechtesten, jedoch zeugt diese Anpassungsgüte trotzdem von einer gelungenen Anpassung.\par
Anhand dieser vier Linien wird nun eine Energiekalibrierung über eine Geradenanpassung durchgeführt, indem den $x$-Koordinaten der Maxima eine
Energie zugeordnet wird. Um die Intensitätsmaxima mit einer Energie zu identifizieren, wird das \textit{X-Ray Data Booklet} \cite{xraydata} verwendet.
Die hier aufgeführten Energien für die entsprechenden Linien sind ohne Unsicherheit angegeben.
Die resultierenden Datenpunkte sind in \cref{tab:callibration} dargestellt.
\begin{table}[H]
   \centering
\caption{Kanäle mit dazugehörigen Energien zur Kallibration der Kanäle}
\begin{tabular}{c c}
\hline Digitaler Kanal $K$ & Energie $E$ / \unit{\electronvolt} \\ 
\hline
$\num{83.57\pm 0.09}$ & $\num{6403.84}$ \\
$\num{88.1\pm 0.2}$ & $\num{7057.98}$ \\
$\num{112.62\pm 0.13}$ & $\num{8638.86}$ \\
$\num{125.7\pm 1.3}$ & $\num{9572.00}$ \\
\hline\end{tabular}
\label{tab:callibration}
\end{table}
An diese Datenpunkte wird nun eine Gerade der Form $E(K) = m \cdot K + b$ angepasst. Die resultierende Anpassungsgerade ist in \cref{fig:kalibrationskurve}
dargestellt. Die Anpassungsgüte (\textit{Residual Variance}) ist mit einem Wert von 187 relativ hoch, was an dem zweiten Messpunkt in liegt.
Dennoch wird die Energiekalibrierung gut genug für die weitere Auswertung sein, da die anderen drei Messpunkte sehr gut durch die Gerade
angepasst werden. Außerdem können gleich einige der berechneten Energien der Referenzspektren mit Literaturwerten aus \cite{xraydata} verglichen werden.
Mithilfe der ermittelten Anpassungsparameter $m = \SI{0,076(6)}{\kilo \electronvolt}$ und $b = \SI{0,045(5)}{\kilo \electronvolt}$ können nun für alle anderen aufgenommenen Spektren digitale Kanäle $K$ in Energien
/ \unit{\electronvolt} umgerechnet werden. Mit
\begin{equation*}
    E = m \cdot K + b , \quad \Delta E = \sqrt{(K \Delta m)^2 + (m \Delta K)^2 + (\Delta b)^2}
\end{equation*} können nun die Positionen der Maxima der Referenzspektren sowie der unbekannten Legierungen in Energien umgerechnet werden.
\begin{figure}[H]
	\centering
	\includegraphics[width=0.6\linewidth]{../figs/kallibrationskurve.pdf}
	\caption{Geradenanpassung zur Energiekalibration. Hierbei sind die meisten Messunsicherheiten zu klein, um auf dem Diagramm 
        sichtbar zu bleiben.}
	\label{fig:kalibrationskurve}
\end{figure} Die ermittelten Energien der Linien der Referenzspektren sind in \cref{tab:energien-charakeristische-linien} eingetragen. Ein Vergleich
dieser experimentell bestimmten Werte für die Energien der charakteristischen Linien mit den entsprechenden Literaturwerten aus \cite{xraydata} zeigt,
dass die Energiekalibrierung hinreichend genau ist. Eine explizite Aufführung des Vergleiches wird hier aus ökonomischen Gründen nicht durchgeführt.


\begin{table}[H]
    \centering
    \caption{Gemessene Energien und Höhen der charakteristischen Linien verschieder Metalle.}
    \label{tab:energien-charakeristische-linien}
    \begin{tabular}{c|c|c}
        Metall & Energie $E$ / \unit{\kilo\electronvolt} & Höhe in Detektionen \\\hline\multirow{4}{*}{FeZn} & \num{6.4\pm 0.7} & \num{10700\pm 600} \\ & \num{6.8\pm 0.8} & \num{4300\pm 300} \\ & \num{8.7\pm 0.9} & \num{1430\pm 80} \\ & \num{9.7\pm 0.9} & \num{200\pm 300} \\\hline
\multirow{5}{*}{Ag} & \num{3.5\pm 0.6} & \num{223\pm 9} \\ & \num{8.2\pm 0.8} & \num{72\pm 17} \\ & \num{9.2\pm 0.9} & \num{33\pm 7} \\ & \num{22.2\pm 1.7} & \num{240\pm 14} \\ & \num{24.9\pm 1.9} & \num{26\pm 3} \\\hline
\multirow{4}{*}{Au} & \num{8.5\pm 0.8} & \num{248\pm 14} \\ & \num{10.0\pm 0.9} & \num{1170\pm 90} \\ & \num{12\pm 1} & \num{640\pm 40} \\ & \num{13.7\pm 1.2} & \num{53\pm 7} \\\hline
\multirow{2}{*}{Cu} & \num{8.2\pm 0.8} & \num{7900\pm 500} \\ & \num{9.0\pm 0.9} & \num{1090\pm 120} \\\hline
\multirow{4}{*}{In} & \num{3.8\pm 0.6} & \num{351\pm 13} \\ & \num{8.3\pm 0.8} & \num{206\pm 19} \\ & \num{9.1\pm 0.9} & \num{43\pm 14} \\ & \num{27\pm 3} & \num{11.8\pm 1.7} \\\hline
\multirow{3}{*}{Fe} & \num{6.4\pm 0.7} & \num{13600\pm 1200} \\ & \num{6.8\pm 0.8} & \num{5200\pm 600} \\ & \num{9.9\pm 0.9} & \num{370\pm 170} \\\hline
\multirow{2}{*}{Mo} & \num{17.7\pm 1.4} & \num{1050\pm 60} \\ & \num{19.8\pm 1.6} & \num{149\pm 9} \\\hline
\multirow{3}{*}{Ni} & \num{6.5\pm 0.7} & \num{310\pm 30} \\ & \num{7.5\pm 0.8} & \num{6500\pm 600} \\ & \num{7.9\pm 0.8} & \num{4720\pm 160} \\\hline
\multirow{5}{*}{Pb} & \num{8.2\pm 0.8} & \num{320\pm 30} \\ & \num{9.2\pm 0.9} & \num{134\pm 16} \\ & \num{11\pm 1} & \num{1210\pm 80} \\ & \num{12.9\pm 1.1} & \num{700\pm 30} \\ & \num{15.1\pm 1.3} & \num{57\pm 5} \\\hline
\multirow{4}{*}{Sn} & \num{4.0\pm 0.6} & \num{640\pm 30} \\ & \num{8.3\pm 0.8} & \num{160\pm 30} \\ & \num{9.0\pm 0.9} & \num{29\pm 8} \\ & \num{25.1\pm 1.9} & \num{58\pm 4} \\\hline
\multirow{2}{*}{Titan} & \num{4.7\pm 0.7} & \num{4900\pm 800} \\ & \num{4.9\pm 0.7} & \num{4700\pm 400} \\\hline
\multirow{5}{*}{W} & \num{5.7\pm 0.7} & \num{50\pm 10} \\ & \num{7.6\pm 0.8} & \num{80\pm 30} \\ & \num{8.6\pm 0.9} & \num{1530\pm 50} \\ & \num{10.0\pm 0.9} & \num{1050\pm 50} \\ & \num{12\pm 1} & \num{101\pm 7} \\\hline
\multirow{2}{*}{Zn} & \num{8.8\pm 0.9} & \num{7200\pm 500} \\ & \num{9.7\pm 0.9} & \num{990\pm 90} \\\hline
\multirow{3}{*}{Zr} & \num{12\pm 1} & \num{56\pm 11} \\ & \num{16.0\pm 1.3} & \num{1690\pm 70} \\ & \num{17.9\pm 1.4} & \num{230\pm 30} \\\hline

    \end{tabular}
\end{table}\newpage

\subsubsection*{Bestimmung der Bestandteile der unbekannten Legierungen}\label{subsubsec:bestandteile}
\textbf{Legierung 1}\newline
Das Röntgenfluoreszenzspektrum der ersten unbekannten Legierung inklusive Gauß-Kurven-Anpassung ist in \cref{fig:unbekannt1} dargestellt. Aus der Anpassung ergeben sich die
in \cref{tab:unbekannt1} zu findenen Anpassungsparameter.
\begin{figure}[H]
	\centering
	\includegraphics[width=0.6\linewidth]{../figs/Unbekannt1.pdf}
	\caption{Röntgenfluoreszenzspektrum der unbekannten Legierung 1.}
	\label{fig:unbekannt1}
\end{figure}
\begin{table}[H]
    \centering
    \caption{Energien der charakteristischen Linien der unbekannten Legierung 1}
    \label{tab:unbekannt1}
    \begin{tabular}{c|c}
       Energie $E$ / \unit{\kilo\electronvolt} & Höhe in Detektionen \\
\hline
\num{5.5\pm 0.7} & \num{2800\pm 400} \\ 
\num{6.5\pm 0.7} & \num{8000\pm 1000} \\ 
\num{6.7\pm 0.8} & \num{3800\pm 300} \\ 

    \end{tabular}
\end{table} Wird der Wert der ersten Linie aus \cref{tab:unbekannt1} mit den Werten der Referenzspektren aus \cref{tab:energien-charakeristische-linien} verglichen,
so findet sich kein übereinstimmender Wert. Dies legt nahe, dass diese charakeristische Linie der unbekannten Legierung auf ein Element zurückzuführen ist,
welches nicht als Referenzelement im Versuch untersucht wurde. Ein Blick in \cite{xraydata} macht deutlich, dass diese Linie auf Chrom (Cr) zurückzuführen ist,
da die $\mathrm{K}_{\alpha_1}$- bzw. $\mathrm{K}_{\alpha_2}$-Linien mit Energien von ungefähr \SI{5,415}{\kilo \electronvolt} bzw. \SI{5,406}{\kilo \electronvolt}
im Rahmen der Unsicherheiten mit der ersten Linie der ersten unbekannten Legierung übereinstimmen. Linie 2 und 3 können leicht mit Linie 1 und 2 von Eisen (Fe)
in \cref{tab:energien-charakeristische-linien} identifiziert werden. Im Rahmen der Unsicherheiten lässt sich diese Zuordnung im Prinzip eindeutig feststellen.
Damit handelt es sich bei Legierung 1 um eine Zusammensetzung aus Chrom und Eisen.\\

\noindent\textbf{Legierung 2}\newline
Das Röntgenfluoreszenzspektrum der zweiten unbekannten Legierung inklusive Gauß-Kurven-Anpassung ist in \cref{fig:unbekannt2} dargestellt. Aus der Anpassung ergeben sich die
in \cref{tab:unbekannt2} zu findenen Anpassungsparameter.\par
Die erste Linie dieser unbekannten Legierung stimmt im Rahmen der Unsicherheiten sehr gut mit der ersten Linie von Kupfer in \cref{tab:energien-charakeristische-linien} überein.
Die Identifikation der zweiten Linie ist nicht ganz eindeutig, da hier sowohl die zweite Linie von Kupfer als auch die erste Linie von Zink (siehe \cref{tab:energien-charakeristische-linien})
in Frage kommt. Jedoch lässt sich die dritte Linie der unbekannten Legierung eindeutig zu der zweiten Linie von Zink zuordnen, weshalb es sich bei der zweiten unbekannten
Legierung um eine Zusammensetzung aus Kupfer und Zink handeln muss.
\begin{figure}[H]
	\centering
	\includegraphics[width=0.6\linewidth]{../figs/Unbekannt2.pdf}
	\caption{Röntgenfluoreszenzspektrum der unbekannten Legierung 2.}
	\label{fig:unbekannt2}
\end{figure}
\begin{table}[H]
    \centering
    \caption{Energien der charakteristischen Linien der unbekannten Legierung 2}
    \label{tab:unbekannt2}
    \begin{tabular}{c|c}
       Energie $E$ / \unit{\kilo\electronvolt} & Höhe in Detektionen \\
\hline
\num{8.1\pm 0.8} & \num{4240\pm 150} \\ 
\num{8.7\pm 0.9} & \num{3610\pm 90} \\ 
\num{9.7\pm 0.9} & \num{140\pm 15} \\ 

    \end{tabular}
\end{table}

\noindent\textbf{Legierung 3}\newline
Das Röntgenfluoreszenzspektrum der dritten unbekannten Legierung inklusive Gauß-Kurven-Anpassung ist in \cref{fig:unbekannt3} dargestellt. Aus der Anpassung ergeben sich die
in \cref{tab:unbekannt3} zu findenen Anpassungsparameter.
\begin{figure}[H]
	\centering
	\includegraphics[width=0.6\linewidth]{../figs/Unbekannt3.pdf}
	\caption{Röntgenfluoreszenzspektrum der unbekannten Legierung 3.}
	\label{fig:unbekannt3}
\end{figure}
\begin{table}[H]
    \centering
    \caption{Energien der charakteristischen Linien der unbekannten Legierung 3}
    \label{tab:unbekannt3}
    \begin{tabular}{c|c}
       Energie $E$ / \unit{\kilo\electronvolt} & Höhe in Detektionen \\
\hline
\num{8.1\pm 0.8} & \num{3600\pm 400} \\ 
\num{8.6\pm 0.9} & \num{2900\pm 100} \\ 
\num{9.8\pm 0.9} & \num{70\pm 30} \\ 

    \end{tabular}
\end{table} Bei der dritten unbekannten Legierung scheint es sich um eine nahezu identische Zusammensetzung wie die zweite unbekannte Legierung zu handeln,
da die Energien der charakteristischen Linien im Rahmen der Unsicherheiten identisch sind. Daher wird auch bei dieser dritten unbekannten Legierung
auf eine Zusammensetzung von Kupfer und Zink geschlossen. Höchstwahrscheinlich handelt es sich jedoch nicht um die identische Legierung, da die Intensitäten
der charakteristischen Linien sich leicht unterscheiden. Daher sind bei Legierung 3 wahrscheinlich die Massenanteile leicht anders als bei Legierung 2.\\

\noindent\textbf{Legierung 4}\newline
Das Röntgenfluoreszenzspektrum der vierten unbekannten Legierung inklusive Gauß-Kurven-Anpassung ist in \cref{fig:unbekannt4} dargestellt. Aus der Anpassung ergeben sich die
in \cref{tab:unbekannt4} zu findenen Anpassungsparameter.
\begin{figure}[H]
	\centering
	\includegraphics[width=0.6\linewidth]{../figs/Unbekannt4.pdf}
	\caption{Röntgenfluoreszenzspektrum der unbekannten Legierung 4.}
	\label{fig:unbekannt4}
\end{figure}
\begin{table}[H]
    \centering
    \caption{Energien der charakteristischen Linien der unbekannten Legierung 4}
    \label{tab:unbekannt4}
    \begin{tabular}{c|c}
       Energie $E$ / \unit{\kilo\electronvolt} & Höhe in Detektionen \\
\hline
\num{5.7\pm 0.7} & \num{520\pm 80} \\ 
\num{5.2\pm 0.7} & \num{200\pm 70} \\ 
\num{7.1\pm 0.8} & \num{4600\pm 400} \\ 
\num{7.8\pm 0.8} & \num{710\pm 170} \\ 

    \end{tabular}
\end{table} Bei dieser unbekannten Legierung ist eine Identifikation der charakteristischen Linien mit denen der Referenzspektren aus \cref{tab:energien-charakeristische-linien}
nicht möglich, da es hier scheinbar keine eindeutige Zuordnung gibt. Zur Identifikation der Linien wird nun die \textit{X-ray Transition Energies Database} \cite{nist_search}
als Hilfswerkzeug benutzt. Hier kann nach den Linien für ein bestimmtes Energie-Intervall gesucht werden. Bei den Linien 1 und 2 der vierten unbekannten Legierung
handelt es sich vermutlich um Neodym, da sich diese mit den Neodym-Linien mit den Energien \SI{5,721}{\kilo \electronvolt} und \SI{5,208}{\kilo \electronvolt}
identifizieren lassen. Die Linien 3 und 4 lassen sich zu den Kobalt-Linien mit den Energien \SI{6,930}{\kilo \electronvolt} und \SI{7,706}{\kilo \electronvolt}
zuordnen. Daher handelt es sich bei dieser unbekannten Legierung vermutlich um eine Zusammensetzung aus Neodym und Kobalt, wobei diese Zuordnung hier
aufgrund fehlender Referenzspektren schwer fällt.
\subsubsection*{Massenanteilsbestimmung der unbekannten Legierung 2}\label{subsubsec:massenanteile}
Gemäß der in \cite{skript} hergeleiteten Formel lassen sich die Massenanteile einer Legierung bestimmen mit 
\begin{equation}
    C_i = \frac{\rho_ih_i}{\sum_j\rho_j h_j}, \qquad \Delta C_i = C_i(1-C_i)\frac{\Delta h_i}{h_i},
\end{equation}
wobei $\rho_i$ die Dichten der einzelnen Elemente sind und $h_i$ die relative Höhe einer Linie 
im Vergleich zum gemessenen Referenzspektrum 
\begin{equation}
    h_i = \frac{H_i}{H_{0, i}},\qquad \Delta h_i = h_i\sqrt{\qty(\frac{\Delta H_i}{H_i})^2 + \qty(\frac{\Delta H_{0, i}}{H_{0, i}})^2},
    \label{eq:rel_height}
\end{equation}
wobei die Höhe dabei die Höhe der angepassten Gauß-Kurve darstellt 
und nicht die gesamte Anzahl der Teilchendetektionen der Linie. Aufgrund der Überlagerung verschiedener Linien 
sind diese Zahlen unterschiedlich. 
Die Höhen der Linien sind Tabellen \ref{tab:energien-charakeristische-linien} sowie \ref{tab:unbekannt2} zu entnehmen.
Hierbei muss jedoch zusätzlich beachtet werden, dass zwei Linien von Kupfer und Zink sich überlagern,
weshalb aus \cref{tab:unbekannt2} die Höhe der ersten Zink-Linie nicht unmittelbar ablesbar ist.\par 
Im Folgenden bezeichnen wir die höhere Kupfer-Linie als $H^1_{0, \mathrm{Cu}}$ und die kleinere als $H^2_{0, \mathrm{Cu}}$.
Das Verhältnis beider Linien $g = H^1_{0, \mathrm{Cu}} / H^2_{0, \mathrm{Cu}}$ mit gaußschem Fehler wie in \ref{eq:rel_height} muss ebenfalls 
in der Legierung unverändert bleiben, weshalb die Höhe der ersten Zink-Linie berechnet werden kann mit 
\begin{equation*}
    H^1_\mathrm{Zn} = H - g\cdot H^1_\mathrm{Cu}, \qquad 
    \Delta H^1_\mathrm{Zn} = \sqrt{(\Delta H)^2 + ( H^1_\mathrm{Cu}\Delta g)^2 + (\Delta  H^1_\mathrm{Cu} g)^2}
\end{equation*}
mit der gesamten Höhe $H$ der Messung. Gemessen wurde das Verhältnis $g=13.8(2)\%$, womit 
der Literaturwert \cite{xraydata} von $17\%$ zwar in der gleichen Größenordnung liegt, jedoch nicht im 
$3\sigma$-Bereich der Messung. Dies könnte darauf hindeuten, dass die hier gemessen Höhen 
nicht mit hoher Genauigkeit repräsentativ sind aufgrund von unerwarteten Abweichungen, die auch dazu führen können,
dass die hier bestimmten Massenanteile von den tatsächlichen Anteilen abweichen.\par
Mit $\rho_\mathrm{Cu} =\SI{8.96}{\gram\per\cubic\cm}$ und $\rho_\mathrm{Zn} =\SI{7.13}{\gram\per\cubic\cm}$ 
\cite{Workman:2022ynf} 
ergeben sich Massenanteile von 
\begin{equation}
    C_\mathrm{Cu} = 49.6(5)\%,\quad C_\mathrm{Zn} = 50(2)\%.
\end{equation}

Damit sind in dieser Kupfer-Zink-Legierung Zink und Kupfer innerhalb der Unsicherheit gleich verteilt. 
Wie schon erwähnt sind die Intensitätsverhältnisse der Linien nicht konform mit den Literaturwerten,
weshalb Fehler in der Berechnung der Massenanteile entsehen können. Eine Legierung aus gleichen Anteilen 
von Kupfer und Zink sind seltern Vorzufinden. Am üblichsten sind Legierung mit etwa $40\%$ Zink, die 
als Messing bekannt sind \cite{wiki:messing}, die hier bestimmten Massenanteile sind dennoch möglich.

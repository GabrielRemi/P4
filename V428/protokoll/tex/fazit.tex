\section{Fazit}\label{sec:fazit}
Erster Versuchsteil\\ \par
Zweiter Versuchsteil\\ \par
Im dritten Versuchsteil wurde die Symmetrie und Gitterstruktur eines NaCl-Kristalls mithilfe des Laue-Verfahrens
untersucht. Dazu wurde der zu untersuchende NaCl-Kristall über \SI{1800}{\second} mit von einer
Molybdän-Röntgenröhre erzeugten Röntgenstrahlung belichtet. Die Röntgenstrahlung wird teilweilse durch den
NaCl-Kristall transmittiert und teilweise an den unterschiedlichen Netzebenenscharen gebeugt. Hinter dem Kristall
wird die transmittierte und bebeugte Röntgenstrahlung mithilfe eines Röntgenfilms nachgewiesen, welcher nach
einer Entwicklung zur Auswertung bereit steht.\par
Ziel der Auswertung war es, die einzelnen Reflexe auf dem Röntgenfilm mit den entsprechenden Millerschen Indizes
zu identifizieren, welche die Orientierung der einzelnen Netzebenenscharen beschreiben. So konnte rausgefunden werden,
welche Netzebenenscharen des NaCl-Kristalls für welchen Reflex verantwortlich waren. Die Zuordnung der Millerschen
Indizes stellte hier eine Herausforderung da, da das Reflexmuster auf dem Röntgenfilm gewölbt war. Dennoch konnte
die Zuordnung unter Berücksichtung der Symmetrie des NaCl-Kristalls (kubische Symmetrie) erfolgreich durchgeführt werden.
Die Ergebnisse sind in \cref{tab:miller2} zu finden.\par
Anschließend konnten mit den Millerschen Indizes der Netzebenenabstand der einzelnen Netzebenen einer Netzebenenschar,
der Glanzwinkel und die für einen Reflex verantwortlichen Wellenlängen (es tragen mehrere Beugungsordnungen bei)
bestimmt werden. Die berechneten Werte sind in \cref{tab:keineahnung} einzusehen.
\section{Fazit}\label{sec:fazit}
Mithilfe dieses Versuchs wurde die Funktionsweise und Bedienung eines Rastertunnelsmikroskops 
erlernt.\\\par
Zunächst wurde die Handhabung des RTMs mit einer Goldprobe getestet, da diese keine hohe 
Präzision in der Vorbereitung und Durchführung benötigt. Wie zuvor 
erwartet konnten hier grobe Strukturen im \emph{constant-current-mode} betrachtet werden, 
die in ihrem Aussehen Wolken oder Kugel ähneln. Aufgrund der hohen elektrischen 
Leitfähigkeit war die Auflösung einzelner Atome nicht möglich.\\\par
Die HOPG-Probe besitzt nach \cref{fig:hopg1} eine kristalline Struktur und erfordert
für deren Auflösung eine höhere Präzision. Diese Struktur ist aufgrund 
der Oberflächenladungsverteilung nicht vollständig auflösbar, da nur jedes zweite Atom sichtbar gemacht 
werden kann. Dies ist in \cref{fig:hopg_rtm_4nm_1_cur} im \emph{constant-height-mode} 
erkennbar. Die Auflösung der Aufnahmen hat die Bestimmung der Gitterkonstanten erschwert und 
damit die Unsicherheit erhöht. Da die Auflösung von der Spitzentopographie sowie dem Abstand 
zur Spitze bestimmt wird, die Probe jedoch ausreichend nah herangefahren werden konnte, 
ist mit hoher Wahrscheinlichkeit ein nicht ausreichend spitzer Platin-Iridium-Draht benutzt worden.\par 
Mit Aufnahme \cref{fig:hopg_rtm_4nm_1_cur} konnte ein Abstand von 
\[d = \SI{380\pm 160}{\pm}\]
gemessen werden, dessen Wert mit Aufnahme \cref{fig:hopg_rtm_4nm_2_cur} bestätigt wird. Damit 
liegt der Referenzwert von $d = \SI{246}{\pm}$ \cite{rtm-leitpfaden} zwar im $1\sigma$-Bereich der 
Messung, jedoch ist trotzdem eine deutliche Abweichung (etwa \SI{55}{\percent}) nach oben hin zu erkennen. Da die Abweichung 
in x-Richtung der Rasterung wesentlich höher als in y-Richtung ausfällt, ist eine mögliche Erklärung 
hierfür, dass durch thermische Ausdehnung die Rasterposition verschoben wurde und somit 
die Spitze die Ladungsverteilung an einer leicht versetzten Stelle gemessen hat. Bei Skalen von einigen
Atomdurchmessern kann dieser Effekt durchaus bemerkbar sein. Eine weitere Erklärung hierfür könnte
eine falsche Eichung der Piezo-Elemente sein, vorallem von dem Element, welches für 
die Verschiebung in x-Richtung zuständig ist. In diesem Falle würde sich das Piezo 
weniger deformieren als von dem RTM vorgesehen. \par
Im Rahmen der Versuchsunsicherheit kann jedoch die Versuchsdurchführung als erfolgreich 
angesehen werden.
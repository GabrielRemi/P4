\section{Einleitung}\label{sec:einleitung}
\pagenumbering{arabic}
Mithilfe eines Rastertunnelmikroskops (RTM) lassen sich Oberflächen von elektrisch leitfähigen Materialien atomar auflösen. Dies ist besonders
nützlich, wenn z.B. die Oberflächenstruktur eines Materials untersucht werden soll. Bei einem RTM wird sich der quantenmechanische
Tunneleffekt\footnote[1]{Die relevanten Kenntnisse zum Tunneleffekt (und Tunnelstrom) sind in Standard-Lehrwerken wie \cite{Demtröder:829119} und \cite{münster} zu finden.} 
zunutze gemacht, indem zwischen abrasternder Messspitze und der zu untersuchenden Oberfläche eine Spannung angelegt wird, sodass
sich die Fermi-Niveaus der Messspitze und der zu untersuchenden Oberfläche gegeneinander verschieben, womit ein Tunnelstrom messbar wird.
Dieser gemessene Tunnelstrom dient dann in zwei verschiedenen Betriebsmodi des RTMs zur Darstellung der Elektronenverteilung an der Oberfläche
des zu untersuchenden Materials, womit Rückschlüsse auf dessen Oberflächenstruktur getroffen werden können.\par
In diesem Versuch soll die Funktionsweise und die technische Umsetzung eines RTMs verstanden werden. Außerdem soll der Umgang mit einem RTM
erlernt werden, indem eine Gold-Probe und eine HOPG-Probe (hochgeordnetes pyrolytisches Graphit) mikroskopiert werden. Hier ist insbesondere das Ziel,
die Oberfläche des HOPGs mit atomarer Auflösung abzubilden, um die Eichung der RTM-Piezos zu überprüfen.
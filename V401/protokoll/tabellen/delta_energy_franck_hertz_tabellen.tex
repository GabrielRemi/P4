\begin{table}[htb]
   \centering
\caption{Übergangsenergien bei verschiedenen Gegenspannungen in eV}
\begin{tabular}{c c c c c}
\hline Maximum & \SI{2.5}{\volt} & \SI{3.0}{\volt} & \SI{3.5}{\volt} & \SI{4.0}{\volt} \\ 
\hline
$\num{1}$ & $\num{4.55\pm 0.04}$ & $\num{4.68\pm 0.04}$ & $\num{4.71\pm 0.05}$ & $\num{4.88\pm 0.09}$ \\
$\num{2}$ & $\num{4.80\pm 0.04}$ & $\num{4.83\pm 0.04}$ & $\num{4.79\pm 0.05}$ & $\num{4.73\pm 0.06}$ \\
$\num{3}$ & $\num{4.94\pm 0.04}$ & $\num{4.98\pm 0.05}$ & $\num{4.96\pm 0.06}$ & $\num{4.97\pm 0.06}$ \\
$\num{4}$ & $\num{5.03\pm 0.06}$ & $\num{5.05\pm 0.07}$ & $\num{5.10\pm 0.08}$ & $\num{5.11\pm 0.09}$ \\
$\num{5}$ & $\num{5.12\pm 0.09}$ & $\num{5.12\pm 0.11}$ & $\num{5.06\pm 0.13}$ & $\num{5.07\pm 0.15}$ \\
$\langle E\rangle$ & $\num{4.9\pm 0.3}$ & $\num{4.9\pm 0.3}$ & $\num{4.9\pm 0.3}$ & $\num{5.0\pm 0.3}$ \\
\hline\end{tabular}
\label{fig:energy_gegen}
\end{table}\begin{table}[htb]
   \centering
\caption{Übergangsenergien bei verschiedenen Temperaturen in eV}
\begin{tabular}{c c c c c}
\hline Maximum & \SI{165}{\celsius} & \SI{172}{\celsius} & \SI{179}{\celsius} & \SI{186}{\celsius} \\ 
\hline
$\num{1}$ & $\num{4.54\pm 0.04}$ & $\num{4.50\pm 0.04}$ & $\num{4.63\pm 0.06}$ & $\num{4.69\pm 0.13}$ \\
$\num{2}$ & $\num{4.70\pm 0.03}$ & $\num{4.70\pm 0.04}$ & $\num{4.46\pm 0.04}$ & $\num{4.32\pm 0.04}$ \\
$\num{3}$ & $\num{4.93\pm 0.04}$ & $\num{4.91\pm 0.04}$ & $\num{4.70\pm 0.04}$ & $\num{4.65\pm 0.04}$ \\
$\num{4}$ & $\num{4.98\pm 0.05}$ & $\num{5.00\pm 0.05}$ & $\num{4.82\pm 0.05}$ & $\num{4.79\pm 0.05}$ \\
$\num{5}$ & $\num{4.95\pm 0.08}$ & $\num{5.09\pm 0.08}$ & $\num{4.78\pm 0.08}$ & $\num{4.73\pm 0.08}$ \\
$\langle E\rangle$ & $\num{4.8\pm 0.3}$ & $\num{4.8\pm 0.3}$ & $\num{4.7\pm 0.2}$ & $\num{4.6\pm 0.3}$ \\
\hline\end{tabular}
\label{fig:energy_temp}
\end{table}
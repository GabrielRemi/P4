% Dokumentenklasse
\documentclass{article}

% Pakete
\usepackage{tex/pakete}
\usepackage{tex/macros}
\DeclareSIUnit\px{px}

% Einstellen der Pakete
\graphicspath{{figs/}}
\addbibresource{refs.bib}
\titleformat{\section}
  {\normalfont\LARGE\bfseries}{\thesection.}{.3em}{}
\titlespacing*{\section}{0pt}{3.5ex plus 1ex minus .2ex}{2.3ex plus .2ex}
\titleformat{\subsection}
  {\normalfont\Large\bfseries}{\thesubsection.}{.3em}{}
\titlespacing*{\subsection}{0pt}{3.5ex plus 1ex minus .2ex}{2.3ex plus .2ex}
\titleformat{\subsubsection}
  {\normalfont\large\bfseries}{\thesubsubsection.}{.3em}{}
\titlespacing*{\subsubsection}{0pt}{3.5ex plus 1ex minus .2ex}{2.3ex plus .2ex}

\sisetup{output-decimal-marker = {,}}

%\numberwithin{equation}{section}
%\counterwithin{figure}{section}

\pagestyle{fancy}
%\fancyhf{}
\lhead{P401 Elektronische Übergänge in Atomen}
\rhead{Gabriel Remiszewski und Christian Fischer}

% Titelblatt
\title{\textbf{Versuchsbericht} \\ \Large{\textbf{P401 Elektronische Übergänge}}}
\date{durchgeführt am 13/14.12.2023 \\ betreut von Valentin Jonas}
\author{Gabriel Remiszewski und Christian Fischer}

%========================================================================
\begin{document}

\begin{titlepage}
\maketitle
\thispagestyle{empty}
\end{titlepage}
\newpage
\tableofcontents
\thispagestyle{empty}
\newpage

% "Einleitung"
\section{Einleitung}\label{sec:einleitung}
\pagenumbering{arabic}

$\displaystyle \pi\mathrm \pi \mu \upmu \muup \uppi \piup \phi\upphi\phiup \int f(\vb* x)\dl^3x$

% "Versuchsabschnitte"
\section{Zeeman-Effekt}\label{sec:zeeman}
Der Zeeman-Effekt beschreibt die Aufspaltung der Energieniveaus einzelner Zustände 
in einem Magnetfeld. Spin und Bahndrehimpuls bewirken ein magnetisches Moment, auf welches 
an ein äußere Magnetfeld wirken kann. Quantenmechanisch kann dies durch den Hamiltonoperator 
beschrieben werden, der die Interaktion mit einem äußeren Feld beschreibt \cite{Sakurai}:
\begin{equation}
    \hat H_\mathrm{int} = -\frac{q}{2m_\mathrm e}\qty(\hat{\vec L} + 2\hat{\vec S})\cdot \vec B.
    \label{eq:zeeman_hamilton}
\end{equation}

In diesem Versuch wird die Aufspaltung an Cadmium beobachtet. Das äußere Elektron sieht 
aufgrund der Wahrscheinlichkeitsverteilung in erster Näherung den Kern mit den inneren 
Schalen effektiv als nur ein Teilchen, womit die Berechnung der Energieniveaus 
äquivalent zum Wasserstoffatom betrachtet werden kann. Da dieses durch eine gerade 
Elektronenzahl keinen Gesamtspin 
besitzt und das Magnetfeld homogen ist, vereinfacht sich \cref{eq:zeeman_hamilton} zu 
\begin{equation*}
    \hat H_\mathrm{int} = \frac{eB}{2m_\mathrm e}\hat{L_z},
\end{equation*}
wobei die Elektronenladung eingesetzt wurde und das Magnetfeld per Konvention in die z-Achse zeigt.
Da bei kugelsymmetrischem Potenzial die z-Komponente des Drehimpulsoperators mit dem restlichen Hamiltonoperator
des Atoms kommutiert, ergibt sich eine Verschiebung des Energieniveaus von
\begin{equation*}
    \Delta E = \frac{e\hbar}{2m_\mathrm e}m_jB \equiv \mu_\mathrm B m_j B
    \label{eq:energy_zeeman}
\end{equation*}

mit dem Bohrschen Magneton \cite{Demtröder:829119}
\begin{equation}
    \mu_B = \SI{9.274015}{\joule\per\tesla}.
    \label{eq:magneton}
\end{equation}
Weil hier elektrische Dipolübergänge betrachtet werden, gilt die Auswahlregel 
$\Delta m = \{\pm 1, 0\}$ \cite{Demtröder:829119}, wobei einer Differenz 
von null linear polarisiertes Licht (im folgenden als $\pi$ bezeichnet) und einer Differenz von $\pm 1$
zirkular polarisiertes Licht ($\sigma^+$ für rechts zirkular und $\sigma^-$ für links zirkular) 
zugeordnet werden kann.
Für $\pi$-Polarisation kann daher keine Energieverschiebung beobachtet werden, für 
$\sigma^\pm$-Polarisation eine Verschiebung von 
\begin{equation}
    \delta E = \pm \mu_\mathrm B B
    \label{eq:normal_zeeman}.
\end{equation}
Unabhängig von der Drehimpulsquantenzahl kann somit beim hier beschriebenen 
normalen Zeeman-Effekt immer nur eine Aufspaltung in drei Linien beobachtet werden,
weshalb die Entartung nur teilweise aufgehoben werden kann.

In \cref{fig:zeeman_übergänge} ist das Übergangsschema für die Übergänge ${}^1D_2\rightarrow {}^1P_1$
des Cadmiumatoms dargestellt. Die Wellenlänge des Übergangs ist $\lambda_0 = \SI{644}{\nano\meter}$
und entspricht rotem Licht. Dieser Übergang wird in diesem Versuch untersucht.

\begin{figure}
    \centering
    \includegraphics[width=0.6\linewidth]{../figs/zeeman_übergänge}
    \caption{Übergangsschema von Cadmium für die Singulett-Zustände ${}^1D_2\rightarrow {}^1P_1$. 
        Zusehen sind jeweils drei Übergangsgruppen mit je drei Übergängen, die untereinander
        entartet sind, die Entartung zwischen einander jedoch durch den Zeeman-Effekt aufgehoben wird. Darunter 
        ist die durch die Auswahlregeln bestimmte Polarisation des abgestrahlten Lichts aufgetragen. 
        \cite{zeeman_handblatt}}
    \label{fig:zeeman_übergänge}
\end{figure}

\subsection{Versuchsaufbau}
Der Aufbau zur Untersuchung des Zeeman-Effekts ist in \cref{fig:zeeman_aufbau} skizziert. Eine 
Cadmiumlampe wird zwischen ein Magnetfeld festgehalten, welches durch zwei in Reihe 
geschalteten stromdurchflossenen Spulen erzeugt wird. Die Spulen können 
um die vertikale Achse gedreht werden, um die Richtung 
des Magnetfeldes im Bezug zur Beobachtungsrichtung verändern zu können. 
Das Licht der Lampe wird mit einer Kondensorlinse 
$f=\SI{150}{\mm}$ kollimiert mit leichter konvergenz, um nachher am Etalon verschieden Einfallswinkel zu 
erzeugen. 

Das eingebaute Fabry-P\'erot-Etalon ist eine planparallele Glasschicht, welche 
beidseitig mit teildurchlässigen Spiegeln beschichtet ist. Durch ständige Reflektion innerhalb der Glasschicht
wird ein Strahlenbündel erzeugt, welches durch Gangunterschiede untereinander interferiert. 
Das hier benutzte Etalon hat einen Brechungsindex von $n=1.457$ und einem Reflexionsgrad $R=0.85$.
Mit einer weiteren Sammellinse $f=\SI{150}{\mm}$ wird das Licht am Okular scharf abgebildet, an 
dem durch die entstandene Interferenz konzentrische Ringe beobachter sein können. Davor wird mit einem 
Interferenzfilter das rote Licht bei $\SI{644}{\nm}$ gefiltert.

\begin{figure}[h]
    \centering
    \includegraphics[width=0.6\linewidth]{../figs/etalon}
    \caption{Funktionsweise eines Fabry-P\'erot-Etalon. $\alpha$ bezeichnet den Einfallswinkel 
        eines Lichtstrahls in den Etalon mit Dicke $d$ bei Brechungsindex $n$.
        $\Delta_1$ und $\Delta_2$ bezeichnen die Gangunterschiede, die zur Interferenz zweier 
        transmittierter Strahlen führen.
        \cite{zeeman_handblatt}}
    \label{fig:etalon}
\end{figure}

\begin{figure}[h]
    \centering
    \includegraphics[width=0.7\linewidth]{../figs/zeeman_aufbau}
    \caption{Aufbau zur Messung des Zeeman-Effekts. \abf a zeigt die Cadmiumlampe mit Klammern
        \abf b und Polschuhe \abf c. \abf d zeigt die Kondensorlinse, \abf e das Etalon, \abf f 
        die Abbildungslinse, \abf g das Interferenzfilter und \abf h das Okular mit Strichskala. 
        \cite{zeeman_handblatt}}
    \label{fig:zeeman_aufbau}
\end{figure}

\subsection{Untersuchung der Transversal- und Longitudinalkonfiguration}
Durch das Drehen der Spulen lässt sich die Polarisation der emittierten Strahlung 
untersuchen. Hierbei werden zwei Konfigurationsmöglichkeiten gesondert betrachtet.
Klassisch lässt sich die Polarisations- und Strahlrichtung der Übergangsstrahlung 
mit dem Lorentz-Modell beschreiben, wo eine Schwingungsgleichung mit der Lorentzkraft als 
treibende Kraft angesehen wird. Dabei sind $\pi$- und $\sigma^\pm$-Strahlung die Eigenmodi 
der Lösung. Diese sind in \cref{fig:zeeman_polarisation} skizziert. Zu sehen ist hierbei, 
dass sich jede Lösung als schwingender Dipol interpretieren lässt, wobei für $\sigma^\pm$ 
zwei senkrecht zueinander stehenden Dipole mit einer $\SI{90}{\degree}$-Phasenbeziehung 
betrachtet werden.

Da Dipole nicht in Bewegungsrichtung abstrahlen, kann in longitudinaler 
Beobachtsrichtung (parallel zum Magnetfeld) nur die zirkular polarisierte $\sigma^\pm$-Strahlung
beobachtet werden. Bei transversaler Konfiguration sind alle drei Modi sichtbar mit dem 
Unterschied, dass $\sigma^\pm$ hier linear polarisiert ist und räumlich um $\SI{90}{\degree}$
zur $\pi$-Polarisation gedreht.

\subsubsection{Transversalkonfiguration}
In Transversalrichtung können $\sigma^\pm$- sowie $\pi$-Strahlung beobachtet werden, wobei 
diese Modi linear polarisiert sind und senkrecht zueinander stehen. Mit einem Polarisationsfilter 
lässt sich somit $\pi$ und $\sigma$ voneinander trennen, $\sigma^+$ und $\sigma^-$ sind jedoch nicht 
unterscheidbar voneinander. Vor das Etalon wird der Polarisationsfilter befestigt und 
beim Einschalten des Magnetfeldes das Bild beobachtet.

\begin{figure}
    \centering
    \begin{subfigure}{0.45\linewidth}
        \centering
        \includegraphics[width=\linewidth]{../figs/transversal_konfig}
        \caption{Transversal Konfiguration: links oben ohne B-Feld, rechts oben mit B-Feld,
            rechts unten mit Filter auf $\SI{0}{\degree}$, links unten mit Filter auf $\SI{90}{\degree}$.}
        \label{fig:transversal_konfiguration}
    \end{subfigure}
    \hspace{.5cm}
    \begin{subfigure}{0.45\linewidth}
        \centering
        \includegraphics[width=\linewidth]{../figs/longitudinal_konfig}
        \caption{Longitudinal Konfiguration: links oben ohne B-Feld, rechts oben mit B-Feld, 
        rechts unten mit Filter auf $\SI{+45}{\degree}$, links unten mit Filter auf $\SI{-45}{\degree}$. }
        \label{fig:longitudinal_konfiguration}
    \end{subfigure}
\end{figure}


\subsubsection{Longitudinalkonfiguration}

\begin{figure}[htb]
    \centering
    \includegraphics[width=0.6\linewidth]{../figs/zeeman_polarisation}
    \caption{Polarisation der verschiedenen Übergänge. $\pi$-Strahlung ist nur in transversaler 
    Richtung beobachtbar, $\sigma^\pm$-Strahlung in transversaler und in longitudinaler 
    Richtung. \cite{zeeman_handblatt}}
    \label{fig:zeeman_polarisation}
\end{figure}
\section{Franck-Hertz-Versuch}\label{sec:franck-hertz}
Der Franck-Hertz-Versuch hat historisch maßgeblich zur Entwicklung 
der Quantenmechanik beigetragen und das damals von Niels Bohr vorgestellte 
Atommodell gestützt. Hierbei werden in einer Röhre Elektronen zu einer Anode hin 
beschleunigt und erzeugen einen messbaren 
Strom. Besitzen diese genügend kinetische Energie, so können 
diese inelastisch mit den Quecksilberatomen in dem Röhrengas stoßen, 
wobei der Energieübertrag zur Anregung atomarer Übergänge genutzt wird. 
Aufgrund der verringerten kinetischen Energie der beschleunigten Elektronen 
kommt es zu einem Stromeinbruch. Die dabei entsehende Spannungskurve wird 
wie in \cref{fig:franck-hertz-spannungskurve} erwartet. 
Aus der Differenz den Abständen der Maxima kann somit die 
Energie des Übergangs bestimmt werden. Primär kann 
aufgrund des deutlich höheren Wirkungsquerschnittes der Übergang 
${}6^1\mathrm S_0\rightarrow 6^1\mathrm P_1$ betrachtet werden.

\subsection{Aufbau und Durchführung}
Um zunächst in einem Quecksilbergas freie Ladungsträger zu erzeugen, wird mit einer 
Heizspannung $U_\mathrm h$ ein Glühdraht erwärmt. In \cref{fig:schaltbild_franck_hertz} 
entspricht dies der Kathode K. Die Elektronen werden mit einer angelegten Beschleunigungsspannung
$U_\mathrm B$ zu einem Gitter G beschleunigt. 
Zwischen Gitter G und Anode A wird eine zusätzliche Gegenspannung $U_G$ angelegt, die Elektronen 
abbremst. Diese beschränkt die kinetische Energie der durch das Gitter kommenden
Elektronen nach unten und verringert so die an der Anode auftreffende Elektronenzahl.
Die Spannungskurve wird viermal bei unterschiedlichen Gegenspannungen und konstanter Temperatur 
gemessen und anschließend viermal bei gleichbleibender Gegenspannung und variabler 
Temperatur.

\subsection{Auswertung}
Die Unsicherheiten werden nach Herstellerangaben auf $\pm 1\%$ inklusive $\pm 0.5\%$ 
des Messbereiches gelegt, die hier bei \SI{30}{\volt} lag \cite{sensor-cassy}.
\begin{table}[htbp]
   \centering
\caption{Anpassparameter der Spannungskurve für verschiedene Gegenspannungen}
\begin{tabular}{c|cc|cc|cc|cc}
\hline & \multicolumn{2}{|c}{\SI{2.5}{\volt}} & \multicolumn{2}{|c}{\SI{3.0}{\volt}} & \multicolumn{2}{|c}{\SI{3.5}{\volt}} & \multicolumn{2}{|c}{\SI{4.0}{\volt}}\\

\hline
Maximum & $x_0$ / \unit{\volt} & $\sigma$ / \unit{\volt} & $x_0$ / \unit{\volt} & $\sigma$ / \unit{\volt} & $x_0$ / \unit{\volt} & $\sigma$ / \unit{\volt} & $x_0$ / \unit{\volt} & $\sigma$ / \unit{\volt} \\ 
\hline
$\num{1}$ & $\num{11.97\pm 0.03}$ & $\num{1.35\pm 0.03}$ & $\num{11.96\pm 0.03}$ & $\num{1.28\pm 0.03}$ & $\num{12.10\pm 0.04}$ & $\num{1.49\pm 0.05}$ & $\num{12.17\pm 0.07}$ & $\num{1.84\pm 0.11}$ \\
$\num{2}$ & $\num{16.52\pm 0.03}$ & $\num{1.031\pm 0.016}$ & $\num{16.64\pm 0.03}$ & $\num{1.00\pm 0.02}$ & $\num{16.81\pm 0.04}$ & $\num{0.98\pm 0.03}$ & $\num{17.04\pm 0.05}$ & $\num{1.01\pm 0.04}$ \\
$\num{3}$ & $\num{21.32\pm 0.03}$ & $\num{1.023\pm 0.016}$ & $\num{21.47\pm 0.03}$ & $\num{0.999\pm 0.018}$ & $\num{21.61\pm 0.04}$ & $\num{0.96\pm 0.03}$ & $\num{21.77\pm 0.04}$ & $\num{0.96\pm 0.03}$ \\
$\num{4}$ & $\num{26.26\pm 0.03}$ & $\num{1.031\pm 0.019}$ & $\num{26.45\pm 0.03}$ & $\num{0.995\pm 0.019}$ & $\num{26.56\pm 0.04}$ & $\num{0.97\pm 0.03}$ & $\num{26.74\pm 0.05}$ & $\num{0.97\pm 0.03}$ \\
$\num{5}$ & $\num{31.29\pm 0.05}$ & $\num{1.10\pm 0.03}$ & $\num{31.50\pm 0.06}$ & $\num{1.04\pm 0.04}$ & $\num{31.66\pm 0.07}$ & $\num{1.02\pm 0.04}$ & $\num{31.85\pm 0.08}$ & $\num{0.97\pm 0.05}$ \\
$\num{6}$ & $\num{36.40\pm 0.07}$ & $\num{1.15\pm 0.06}$ & $\num{36.61\pm 0.09}$ & $\num{1.09\pm 0.06}$ & $\num{36.72\pm 0.11}$ & $\num{1.03\pm 0.07}$ & $\num{36.92\pm 0.13}$ & $\num{0.97\pm 0.09}$ \\
\hline\end{tabular}
\label{tab:gegenspannung}
\end{table}\begin{table}[htbp]
   \centering
\caption{Anpassparameter der Spannungskurve für verschiedene Temperaturen}
\begin{tabular}{c|cc|cc|cc|cc}
\hline & \multicolumn{2}{|c}{\SI{165}{\volt}} & \multicolumn{2}{|c}{\SI{172}{\volt}} & \multicolumn{2}{|c}{\SI{179}{\volt}} & \multicolumn{2}{|c}{\SI{186}{\volt}}\\

\hline
Maximum & $x_0$ / \unit{\volt} & $\sigma$ / \unit{\volt} & $x_0$ / \unit{\volt} & $\sigma$ / \unit{\volt} & $x_0$ / \unit{\volt} & $\sigma$ / \unit{\volt} & $x_0$ / \unit{\volt} & $\sigma$ / \unit{\volt} \\ 
\hline
$\num{1}$ & $\num{12.00\pm 0.03}$ & $\num{1.43\pm 0.04}$ & $\num{12.33\pm 0.04}$ & $\num{1.57\pm 0.04}$ & $\num{12.42\pm 0.06}$ & $\num{1.99\pm 0.08}$ & $\num{12.79\pm 0.12}$ & $\num{3.0\pm 0.3}$ \\
$\num{2}$ & $\num{16.55\pm 0.03}$ & $\num{1.023\pm 0.016}$ & $\num{16.82\pm 0.03}$ & $\num{1.109\pm 0.018}$ & $\num{17.05\pm 0.03}$ & $\num{1.21\pm 0.03}$ & $\num{17.48\pm 0.04}$ & $\num{1.21\pm 0.05}$ \\
$\num{3}$ & $\num{21.24\pm 0.03}$ & $\num{1.030\pm 0.014}$ & $\num{21.53\pm 0.03}$ & $\num{1.094\pm 0.016}$ & $\num{21.51\pm 0.03}$ & $\num{1.15\pm 0.02}$ & $\num{21.80\pm 0.03}$ & $\num{1.25\pm 0.03}$ \\
$\num{4}$ & $\num{26.17\pm 0.03}$ & $\num{1.021\pm 0.016}$ & $\num{26.43\pm 0.03}$ & $\num{1.083\pm 0.016}$ & $\num{26.21\pm 0.03}$ & $\num{1.085\pm 0.017}$ & $\num{26.45\pm 0.03}$ & $\num{1.16\pm 0.02}$ \\
$\num{5}$ & $\num{31.15\pm 0.05}$ & $\num{1.06\pm 0.03}$ & $\num{31.43\pm 0.05}$ & $\num{1.14\pm 0.03}$ & $\num{31.04\pm 0.05}$ & $\num{1.06\pm 0.03}$ & $\num{31.24\pm 0.04}$ & $\num{1.11\pm 0.03}$ \\
$\num{6}$ & $\num{36.09\pm 0.06}$ & $\num{1.08\pm 0.04}$ & $\num{36.52\pm 0.07}$ & $\num{1.19\pm 0.05}$ & $\num{35.82\pm 0.06}$ & $\num{1.01\pm 0.05}$ & $\num{35.98\pm 0.07}$ & $\num{1.03\pm 0.05}$ \\
\hline\end{tabular}
\label{tab:temperatur}
\end{table}

\begin{figure}[htb]
    \centering
    \begin{minipage}{.45\linewidth}
        \centering
        \includegraphics[width=\linewidth]{../figs/Franck-Hertz_spannungskurve}
        \caption{Franck-Hertz: Spannungskurve}
        \label{fig:franck-hertz-spannungskurve}
    \end{minipage}
    \hspace{.5cm}
    \begin{minipage}{.45\linewidth}
        \centering
        \includegraphics[width=\linewidth]{../figs/Schaltbild_Franck_Hertz_Versuch}
        \caption{Schaltbild Franck Hertz Versuche}
        \label{fig:schaltbild_franck_hertz}
    \end{minipage}
\end{figure}

\begin{figure}[htb]
    \centering
    \includegraphics[width=0.6\linewidth]{../figs/franck-hertz_gegenspannung}
    \caption{Spannungskurve in Abhängigkeit der Gegenspannung}
    \label{fig:gegenspannung}
\end{figure}

\begin{figure}[htb]
    \centering
    \includegraphics[width=0.6\linewidth]{../figs/franck-hertz_temperatur}
    \caption{Spannungskurve in Abhängigkeit der Temperatur}
    \label{fig:temperatur}
\end{figure}

% "Fazit"
\section{Fazit}\label{sec:fazit}
Mithilfe dieses Versuchs wurde die Funktionsweise und Bedienung eines Rastertunnelsmikroskops 
erlernt.\\\par
Zunächst wurde die Handhabung des RTMs mit einer Goldprobe getestet, da diese keine hohe 
Präzision in der Vorbereitung und Durchführung benötigt. Wie zuvor 
erwartet konnten hier grobe Strukturen im \emph{constant-current-mode} betrachtet werden, 
die in ihrem Aussehen Wolken oder Kugel ähneln. Aufgrund der hohen elektrischen 
Leitfähigkeit war die Auflösung einzelner Atome nicht möglich.\\\par
Die HOPG-Probe besitzt nach \cref{fig:hopg1} eine kristalline Struktur und erfordert
für deren Auflösung eine höhere Präzision. Diese Struktur ist aufgrund 
der Oberflächenladungsverteilung nicht auflösbar, da nur jedes zweite Atom sichtbar gemacht 
werden kann. Dies ist in \cref{fig:hopg_rtm_4nm_1_cur} im \emph{constant-height-mode} 
erkennbar. Die Auflösung der Aufnahmen hat die Bestimmung der Gitterkonstanten erschwert und 
damit die Unsicherheit erhöht. Da die Auflösung von der Spitzentopographie sowie dem Abstand 
zur Spitze bestimmt wird, die Probe jedoch ausreichend nah herangefahren werden konnte, 
ist mit hoher Wahrscheinlichkeit ein nicht ausreichend spitzer Platin-Iridium-Draht benutzt worden.\par 
Mit Aufnahme \cref{fig:hopg_rtm_4nm_1_cur} konnte ein Abstand von 
\[d = \SI{380\pm 160}{\pm}\]
gemessen werden, dessen Wert mit Aufnahme \cref{fig:hopg_rtm_4nm_2_cur} bestätigt wird. Damit 
liegt der Referenzwert von $d = \SI{246}{\pm}$ \cite{rtm-leitpfaden} zwar im $1\sigma$-Bereich der 
Messung, jedoch ist trotzdem eine deutliche Abweichung (etwa \SI{55}{\percent}) nach oben hin zu erkennen. Da die Abweichung 
in x-Richtung der Rasterung wesentlich höher als in y-Richtung ausfällt, ist eine mögliche Erklärung 
hierfür, dass durch thermische Ausdehnung die Rasterposition verschoben wurde und somit 
die Spitze die Ladungsverteilung an einer leicht versetzten Stelle gemessen hat. Bei Skalen von einigen
Atomdurchmessern kann dieser Effekt durchaus bemerkbar sein. Eine weitere Erklärung hierfür könnte
eine falsche Eichung der Piezo-Elemente sein, vorallem von dem Element, welches für 
die Verschiebung in x-Richtung zuständig ist. In diesem Falle würde sich das Piezo 
weniger deformieren als von dem RTM vorgesehen. \par
Im Rahmen der Versuchsunsicherheit kann jedoch die Versuchsdurchführung als erfolgreich 
angesehen werden.

% "Anhang"
\appendix
\section{Anhang}\label{sec:anhang}
\subsubsection*{Darstellung der vom Assistenten bereitgestellten Rohdaten zum ersten Versuchsteil}
\begin{figure}[H]
	\centering
	\includegraphics[width=0.4\linewidth]{../figs/referenz.pdf}
	\caption{Vom Assistenten bereitgestellte Rohdaten der Referenzmessung für ein unbeschichtetes Deckglas für TM- und TE-Polarisation (die Winkelunsicherheiten sind für diesen Winkelbereich zu klein, um erkennbar zu sein).}
	\label{fig:referenz}
\end{figure}
\begin{figure}[H]
    \centering
    \begin{subfigure}{0.4\textwidth}
        \centering
        \includegraphics[width=\linewidth]{../figs/au1}
        \caption{Goldfilm 1}
    \end{subfigure}
    \begin{subfigure}{0.4\textwidth}
        \centering
        \includegraphics[width=\linewidth]{../figs/au2}
        \caption{Goldfilm 2}
    \end{subfigure}
    \begin{subfigure}{0.4\textwidth}
        \centering
        \includegraphics[width=\linewidth]{../figs/au3}
        \caption{Goldfilm 3}
    \end{subfigure}
    \begin{subfigure}{0.4\textwidth}
        \centering
        \includegraphics[width=\linewidth]{../figs/au4}
        \caption{Goldfilm 4}
    \end{subfigure}
    \caption{Vom Assistent bereitgestellte Rohdaten der Reflexionsmessungen an vier Goldfilmen für TM- und TE-Polarisation (die Winkelunsicherheiten sind für diesen Winkelbereich zu klein, um erkennbar zu sein).}\label{fig:gold}
\end{figure}


% Literaturverzeichnis ausgeben
\printbibliography[heading=bibintoc, title = {Literaturverzeichnis}]

\end{document}
%========================================================================
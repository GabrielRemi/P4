\section{Fazit}\label{sec:fazit}
In diesem Versuch wurde zunächst der normale Zeeman-Effekt untersucht und durch Diesen 
das Bohrsche Magneton experimentell bestimmt. Da im normalen Zeeman-Effekt drei 
erlaubte Emissionsmodi mit unterschiedlichen Polarisationen existieren, 
wurde zunächst die Polarisation dieser Modi untersucht. Hierfür konnten mit 
einem Polarisationsfilter und einer Verzögerungsplatte bei transversaler und 
longitudinaler Magnetfeldausrichtung mit den erstellten Bilden in 
\cref{fig:transversal_konfiguration,fig:longitudinal_konfiguration}
gezeigt werden, dass ein durch das Etalon erzeugter Interferenzring sich in
drei einzelne aufspaltet, wobei der mittlere Ring $\pi$-Strahlung zugewiesen werden kann,
der innerer Ring $\sigma^-$ und das äußere $\sigma^+$-Licht.
Mit einer CCD-Kamera konnte die Aufspaltung eines Maximums nahe dem Zentrum gemessen und damit 
das Bohrsche Magneton bestimmt werden. Da beidseitig Aufspaltungen beobachtet wurden, konnten 
zwei Werte bestimmt werden, deren Mittelwert bei 
\begin{equation*}
    \mu_\mathrm B = \SI{9.3(8)e-24}{\joule\per\tesla}
\end{equation*}
liegt, was im Bereich des Literaturwertes $\mu_\mathrm B = \SI{9.274e-24}{\joule\per\tesla}$ liegt.
Die Finesse des Etalons konnte für beide Magnetfeldausrichtungen auf 
\[\mathcal F_\mathrm{trans} = \num{7.5\pm.3},\qquad \mathcal F_\mathrm{longi} = \num{10.4\pm1.7}\]
bestimmt werden, was deutlich kleiner als der Referenzwert $\mathcal F = \num{19.3}$ ist. Hinzu kommt, dass
eine Linienbreite von \SI{3.3(1)}{\pm} anstatt der durch die Dopplerverbreiterung erwarteten \SI{1.4}{\pm}
bestimmt wurden. Daraus wird geschlossen, dass die Sammellinse im Aufbau nicht gut genug fokussiert war.\\

Anschließend wurde der Franck-Hertz-Versuch durchgeführt und durch Stoßanregung die Energie 
eines atomaren Übergangs bestimmt. Wie erwartet konnte eine Energie von \SI{4.9(3)}{\electronvolt}
bestimmt werden. Diese Energie ist unabhängig von der angelegten Gegenspannung, jedoch sinkt diese 
bei Erhöhung der Temperatur. Der Grund hierfür ist die verringerte freie Weglänge, wodurch die 
Elektronen nur auf geringere Geschwindigkeiten beschleunigt werden können und der Wirkungsquerschnitt eines 
niedrigeren Energieniveaus größer als des betrachteten ist und somit die beobachtete Energiedifferenzen der Maxima
der Spannungskurven verschoben werden.

Zusammenfassend kann also gesagt werden, dass die Versuche die erwarteten Ergebnisse liefern und somit der 
Versuch mit Erfolg abgeschlossen wurde.


\section{Franck-Hertz-Versuch}\label{sec:franck-hertz}
Der Franck-Hertz-Versuch hat historisch maßgeblich zur Entwicklung 
der Quantenmechanik beigetragen und das damals von Niels Bohr vorgestellte 
Atommodell gestützt. Hierbei werden in einer Röhre Elektronen zu einer Anode hin 
beschleunigt und erzeugen einen messbaren 
Strom. Besitzen diese genügend kinetische Energie, so können 
diese inelastisch mit den Quecksilberatomen in dem Röhrengas stoßen, 
wobei der Energieübertrag zur Anregung atomarer Übergänge genutzt wird. 
Aufgrund der verringerten kinetischen Energie der beschleunigten Elektronen 
kommt es zu einem Stromeinbruch. Die dabei entsehende Spannungskurve wird 
wie in \cref{fig:franck-hertz-spannungskurve} erwartet. 
Aus der Differenz den Abständen der Maxima kann somit die 
Energie des Übergangs bestimmt werden. Primär kann 
aufgrund des deutlich höheren Wirkungsquerschnittes der Übergang 
${}6^1\mathrm S_0\rightarrow 6^1\mathrm P_1$ betrachtet werden.

\subsection{Aufbau und Durchführung}
Um zunächst in einem Quecksilbergas freie Ladungsträger zu erzeugen, wird mit einer 
Heizspannung $U_\mathrm h$ ein Glühdraht erwärmt. In \cref{fig:schaltbild_franck_hertz} 
entspricht dies der Kathode K. Die Elektronen werden mit einer angelegten Beschleunigungsspannung
$U_\mathrm B$ zu einem Gitter G beschleunigt. 
Zwischen Gitter G und Anode A wird eine zusätzliche Gegenspannung $U_G$ angelegt, die Elektronen 
abbremst. Diese beschränkt die kinetische Energie der durch das Gitter kommenden
Elektronen nach unten und verringert so die an der Anode auftreffende Elektronenzahl.
Die Spannungskurve wird viermal bei unterschiedlichen Gegenspannungen und konstanter Temperatur 
gemessen und anschließend viermal bei gleichbleibender Gegenspannung und variabler 
Temperatur.

\subsection{Auswertung}
Die Unsicherheiten werden nach Herstellerangaben auf $\pm 1\%$ inklusive $\pm 0.5\%$ 
des Messbereiches gelegt, die hier bei \SI{30}{\volt} lag \cite{sensor-cassy}.
\begin{table}[htbp]
   \centering
\caption{Anpassparameter der Spannungskurve für verschiedene Gegenspannungen}
\begin{tabular}{c|cc|cc|cc|cc}
\hline & \multicolumn{2}{|c}{\SI{2.5}{\volt}} & \multicolumn{2}{|c}{\SI{3.0}{\volt}} & \multicolumn{2}{|c}{\SI{3.5}{\volt}} & \multicolumn{2}{|c}{\SI{4.0}{\volt}}\\

\hline
Maximum & $x_0$ / \unit{\volt} & $\sigma$ / \unit{\volt} & $x_0$ / \unit{\volt} & $\sigma$ / \unit{\volt} & $x_0$ / \unit{\volt} & $\sigma$ / \unit{\volt} & $x_0$ / \unit{\volt} & $\sigma$ / \unit{\volt} \\ 
\hline
$\num{1}$ & $\num{11.97\pm 0.03}$ & $\num{1.35\pm 0.03}$ & $\num{11.96\pm 0.03}$ & $\num{1.28\pm 0.03}$ & $\num{12.10\pm 0.04}$ & $\num{1.49\pm 0.05}$ & $\num{12.17\pm 0.07}$ & $\num{1.84\pm 0.11}$ \\
$\num{2}$ & $\num{16.52\pm 0.03}$ & $\num{1.031\pm 0.016}$ & $\num{16.64\pm 0.03}$ & $\num{1.00\pm 0.02}$ & $\num{16.81\pm 0.04}$ & $\num{0.98\pm 0.03}$ & $\num{17.04\pm 0.05}$ & $\num{1.01\pm 0.04}$ \\
$\num{3}$ & $\num{21.32\pm 0.03}$ & $\num{1.023\pm 0.016}$ & $\num{21.47\pm 0.03}$ & $\num{0.999\pm 0.018}$ & $\num{21.61\pm 0.04}$ & $\num{0.96\pm 0.03}$ & $\num{21.77\pm 0.04}$ & $\num{0.96\pm 0.03}$ \\
$\num{4}$ & $\num{26.26\pm 0.03}$ & $\num{1.031\pm 0.019}$ & $\num{26.45\pm 0.03}$ & $\num{0.995\pm 0.019}$ & $\num{26.56\pm 0.04}$ & $\num{0.97\pm 0.03}$ & $\num{26.74\pm 0.05}$ & $\num{0.97\pm 0.03}$ \\
$\num{5}$ & $\num{31.29\pm 0.05}$ & $\num{1.10\pm 0.03}$ & $\num{31.50\pm 0.06}$ & $\num{1.04\pm 0.04}$ & $\num{31.66\pm 0.07}$ & $\num{1.02\pm 0.04}$ & $\num{31.85\pm 0.08}$ & $\num{0.97\pm 0.05}$ \\
$\num{6}$ & $\num{36.40\pm 0.07}$ & $\num{1.15\pm 0.06}$ & $\num{36.61\pm 0.09}$ & $\num{1.09\pm 0.06}$ & $\num{36.72\pm 0.11}$ & $\num{1.03\pm 0.07}$ & $\num{36.92\pm 0.13}$ & $\num{0.97\pm 0.09}$ \\
\hline\end{tabular}
\label{tab:gegenspannung}
\end{table}\begin{table}[htbp]
   \centering
\caption{Anpassparameter der Spannungskurve für verschiedene Temperaturen}
\begin{tabular}{c|cc|cc|cc|cc}
\hline & \multicolumn{2}{|c}{\SI{165}{\volt}} & \multicolumn{2}{|c}{\SI{172}{\volt}} & \multicolumn{2}{|c}{\SI{179}{\volt}} & \multicolumn{2}{|c}{\SI{186}{\volt}}\\

\hline
Maximum & $x_0$ / \unit{\volt} & $\sigma$ / \unit{\volt} & $x_0$ / \unit{\volt} & $\sigma$ / \unit{\volt} & $x_0$ / \unit{\volt} & $\sigma$ / \unit{\volt} & $x_0$ / \unit{\volt} & $\sigma$ / \unit{\volt} \\ 
\hline
$\num{1}$ & $\num{12.00\pm 0.03}$ & $\num{1.43\pm 0.04}$ & $\num{12.33\pm 0.04}$ & $\num{1.57\pm 0.04}$ & $\num{12.42\pm 0.06}$ & $\num{1.99\pm 0.08}$ & $\num{12.79\pm 0.12}$ & $\num{3.0\pm 0.3}$ \\
$\num{2}$ & $\num{16.55\pm 0.03}$ & $\num{1.023\pm 0.016}$ & $\num{16.82\pm 0.03}$ & $\num{1.109\pm 0.018}$ & $\num{17.05\pm 0.03}$ & $\num{1.21\pm 0.03}$ & $\num{17.48\pm 0.04}$ & $\num{1.21\pm 0.05}$ \\
$\num{3}$ & $\num{21.24\pm 0.03}$ & $\num{1.030\pm 0.014}$ & $\num{21.53\pm 0.03}$ & $\num{1.094\pm 0.016}$ & $\num{21.51\pm 0.03}$ & $\num{1.15\pm 0.02}$ & $\num{21.80\pm 0.03}$ & $\num{1.25\pm 0.03}$ \\
$\num{4}$ & $\num{26.17\pm 0.03}$ & $\num{1.021\pm 0.016}$ & $\num{26.43\pm 0.03}$ & $\num{1.083\pm 0.016}$ & $\num{26.21\pm 0.03}$ & $\num{1.085\pm 0.017}$ & $\num{26.45\pm 0.03}$ & $\num{1.16\pm 0.02}$ \\
$\num{5}$ & $\num{31.15\pm 0.05}$ & $\num{1.06\pm 0.03}$ & $\num{31.43\pm 0.05}$ & $\num{1.14\pm 0.03}$ & $\num{31.04\pm 0.05}$ & $\num{1.06\pm 0.03}$ & $\num{31.24\pm 0.04}$ & $\num{1.11\pm 0.03}$ \\
$\num{6}$ & $\num{36.09\pm 0.06}$ & $\num{1.08\pm 0.04}$ & $\num{36.52\pm 0.07}$ & $\num{1.19\pm 0.05}$ & $\num{35.82\pm 0.06}$ & $\num{1.01\pm 0.05}$ & $\num{35.98\pm 0.07}$ & $\num{1.03\pm 0.05}$ \\
\hline\end{tabular}
\label{tab:temperatur}
\end{table}

\begin{figure}[htb]
    \centering
    \begin{minipage}{.45\linewidth}
        \centering
        \includegraphics[width=\linewidth]{../figs/Franck-Hertz_spannungskurve}
        \caption{Franck-Hertz: Spannungskurve}
        \label{fig:franck-hertz-spannungskurve}
    \end{minipage}
    \hspace{.5cm}
    \begin{minipage}{.45\linewidth}
        \centering
        \includegraphics[width=\linewidth]{../figs/Schaltbild_Franck_Hertz_Versuch}
        \caption{Schaltbild Franck Hertz Versuche}
        \label{fig:schaltbild_franck_hertz}
    \end{minipage}
\end{figure}

\begin{figure}[htb]
    \centering
    \includegraphics[width=0.6\linewidth]{../figs/franck-hertz_gegenspannung}
    \caption{Spannungskurve in Abhängigkeit der Gegenspannung}
    \label{fig:gegenspannung}
\end{figure}

\begin{figure}[htb]
    \centering
    \includegraphics[width=0.6\linewidth]{../figs/franck-hertz_temperatur}
    \caption{Spannungskurve in Abhängigkeit der Temperatur}
    \label{fig:temperatur}
\end{figure}
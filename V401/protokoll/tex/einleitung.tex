\section{Einleitung}\label{sec:einleitung}
\pagenumbering{arabic}

In diesem Versuch wird die Quantelung von Energie mit historisch durchgeführten Experimenten untersucht.
Zunächst wird der normale Zeeman-Effekt zum Gebrauch gemacht, um atomare Übergänge durch ein homogenes Magnetfeld
aufzuspalten, was durch eine partielle Aufhebung der Magnetquantenzahl Entartung hervorgerufen wird. Aus
der Energiedifferenz der Übergänge kann das Bohrsche Magneton bestimmt werden.\\
Anschließend wird der Franck-Hertz-Versuch durchgeführt, mit dem durch
Stoßanregung die Energie eines Übergangs bestimmt werden kann.

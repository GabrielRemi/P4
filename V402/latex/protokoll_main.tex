% Dokumentenklasse
\documentclass{article}

% Pakete
\usepackage{tex/pakete}

% Einstellen der Pakete
\graphicspath{{figs/}}
\addbibresource{refs.bib}
\titleformat{\section}
  {\normalfont\LARGE\bfseries}{\thesection.}{.3em}{}
\titlespacing*{\section}{0pt}{3.5ex plus 1ex minus .2ex}{2.3ex plus .2ex}
\titleformat{\subsection}
  {\normalfont\Large\bfseries}{\thesubsection.}{.3em}{}
\titlespacing*{\subsection}{0pt}{3.5ex plus 1ex minus .2ex}{2.3ex plus .2ex}
\titleformat{\subsubsection}
  {\normalfont\large\bfseries}{\thesubsubsection.}{.3em}{}
\titlespacing*{\subsubsection}{0pt}{3.5ex plus 1ex minus .2ex}{2.3ex plus .2ex}

\pagestyle{fancy}
%\fancyhf{}
\lhead{P402 Quantelung von Energie}
\rhead{Gabriel Remiszewski und Christian Fischer}

% Titelblatt
\title{\textbf{Versuchsbericht} \\ \Large{\textbf{P402 Quantelung von Energie}}}
\date{durchgeführt am 18/19.10.2023}
\author{Gabriel Remiszewski und Christian Fischer}

%========================================================================
\begin{document}

\begin{titlepage}
\maketitle
\thispagestyle{empty}
\end{titlepage}
\newpage
\tableofcontents
\thispagestyle{empty}
\newpage

% "Einleitung"
\section{Einleitung}\label{sec:einleitung}
\pagenumbering{arabic}
\cite{skript}

% "Versuchsabschnitte"
\section{Versuchsteil}\label{sec:versuchsteil}

% "Fazit"
\section{Fazit}\label{sec:fazit}
Mithilfe der Bragg-Reflexion konnte von zwei unterschiedlichen Anoden einer 
Röntgenröhre die K$_\alpha$ und K$_\beta$ Linien der charakteristischen 
Röntenstrahlung untersucht werden. Die Wellenlängen multipliziert mit der Beugungsordnung 
der Anode aus einem zuvor unbekannten Element sind in \cref{tab:anode-unbekannt}
vorzufinden. Dabei konnte festgestellt werden, dass sämtliche 
Linien Vielfache der ersten beiden Linien (vom Diagramm aus links gesehen)
darstellen, womit schlussgefolgert werden kann, dass die hier beobachteten Linien
zwei charakeristische Linien bis zur vierten Beugungsordnung darstellen solllen.
Das unbekannte Element konnte als Silber identifiziert werden, da 
dessen Spektrum eine K$_\beta$-Linie bei \SI{49.3}{\pm} aufweist und der 
Messwert von \SI{49.26\pm0.06}{\pm} im $1\sigma$-Bereich der Messung liegt. Eine K$_\alpha$ 
Linie liegt bei \SI{55.942}{\pm}, während eine Wellenlänge von \SI{55.63\pm0.05}{\pm}
gemessen werden konnte. Da die erste Linie mit der Literatur übereinstimmt, ist eine 
Abweichung bei der zweiten Linien ungewöhnlich. Eine Verschiebung 
des Kristalls während der Messung könnte Ursache für die Diskrepanz sein.\par 
Von einer Molybdän-Anode wurde zusätzlich die Feinstrukturaufspaltung 
der K$_\alpha$-Linie in der vierten Beugungsordnung gemessen, was in \cref{fig:feinstruktur}
zu sehen ist. In \cref{tab:feinstruktur} sind die gemessenen 
Energien im Vergleich zu den Literaturwerten zu sehen. Dabei konnte die Feinstrukturaufspaltung
auf drei signifikante Stellen genau gemessen werden, da jedoch die in diesem Versuch 
abgeschätzten Fehler deutlich kleiner sind, liegen die Referenzwerte nicht im $3\sigma$-Bereich
der Messung. Dies ist auch für Wellenlängendifferenz
\begin{equation*}
	\Delta\lambda_\mathrm{Exp} = \SI{0.444\pm 0.008}{\pm},\qquad
	\Delta\lambda_\mathrm{Ref} = \SI{0.428\pm0.004}{\pm}
\end{equation*}
zu beobachten, weshalb ein systematischer Fehler als Ursache für die Abweichung 
ausgeschlossen wird. Eine leichte Verschiebung des Kristalls stellt dabei 
eine realistische Fehlerquelle dar.\\\par

Mit der Röntgenfluoreszenz konnte die Zusammensetzung von 
Legierungen analysiert werden. Hierfür wurden mit verschiedenen metallischen Plätchen 
das dazugehörige Röntegenspektrum durch Fluoreszenz gemessen werden. Nach Kallibration
der Messung durch den Vielkanalanalysator konnte dem Spektrum Energien zugegeordnet werden und die 
Intensitäten der Linien konnten mit den Intensitäten der Legierungen verglichen werden und somit
die Massenanteile der Elemente. Hierbei konnte die zweite Legierung als 
Mischung von Kupfer und Zink identifiziert werden mit, im Fehlerbereich, gleichen Anteilen von 
\begin{equation*}
    C_\mathrm{Cu} = 49.6(5)\%,\quad C_\mathrm{Zn} = 50(2)\%.
\end{equation*}
Einen Vergleichswert hierzu gibt es nicht, weshalb eine quantitative Einordnung der Messwerte 
nicht möglich ist. Der Wert liegt jedoch in realistischen Bereichen.
\\\par

Im dritten Versuchsteil wurde die Symmetrie und Gitterstruktur eines NaCl-Kristalls mithilfe des Laue-Verfahrens
untersucht. Dazu wurde der zu untersuchende NaCl-Kristall über \SI{1800}{\second} mit von einer
Molybdän-Röntgenröhre erzeugten Röntgenstrahlung belichtet. Die Röntgenstrahlung wird teilweilse durch den
NaCl-Kristall transmittiert und teilweise an den unterschiedlichen Netzebenenscharen gebeugt. Hinter dem Kristall
wird die transmittierte und gebeugte Röntgenstrahlung mithilfe eines Röntgenfilms nachgewiesen, welcher nach
einer Entwicklung zur Auswertung bereit steht.\par
Ziel der Auswertung war es, die einzelnen Reflexe auf dem Röntgenfilm mit den entsprechenden Millerschen Indizes
zu identifizieren, welche die Orientierung der einzelnen Netzebenenscharen beschreiben. So konnte rausgefunden werden,
welche Netzebenenscharen des NaCl-Kristalls für welchen Reflex verantwortlich waren. Die Zuordnung der Millerschen
Indizes stellte hier eine Herausforderung da, da das Reflexmuster auf dem Röntgenfilm gewölbt war. Dennoch konnte
die Zuordnung unter Berücksichtung der Symmetrie des NaCl-Kristalls (kubische Symmetrie) erfolgreich durchgeführt werden.
Die Ergebnisse sind in \cref{tab:miller2} zu finden.\par
Anschließend konnten mit den Millerschen Indizes der Netzebenenabstand der einzelnen Netzebenen einer Netzebenenschar,
der Glanzwinkel und die für einen Reflex verantwortlichen Wellenlängen (es tragen mehrere Beugungsordnungen bei)
bestimmt werden. Die berechneten Werte sind in \cref{tab:keineahnung} einzusehen.

% "Anhang"
\appendix
\section*{Anhang}\label{sec:anhang}


% Literaturverzeichnis ausgeben
\printbibliography[heading=bibintoc, title = {Literaturverzeichnis}]

\end{document}
%========================================================================
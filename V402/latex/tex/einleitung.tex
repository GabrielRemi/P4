\section{Einleitung}\label{sec:einleitung}
\pagenumbering{arabic}

In diesem Versuch wird das Plancksche Wirkungsquantum mithilfe des Photoeffekts
sowie mit der Balmer-Serie atomarer Übergänge des Wasserstoff bestimmt.\par
Im ersten Versuchsteil wird eine Photozelle mit Licht unterschiedlicher Frequenzen 
bestrahlt, wo durch den Photoeffekt Elektronen aus einer Kathode 
befreit werden und dies durch einen elektrischen Strom 
nachweisbar gemacht wird.
Aus der Beziehung $E=\mathrm h\nu$ für Photonen kann das Wirkungsquantum bestimmt werden.\par
Im zweiten Versuchsteil wird das Lichtspektrum einer Wasserstofflampe mit einem
Reflexionsgitter spektroskopisch untersucht, welches durch an den Atomkern gebundene und
angeregte Elektronen entsteht, die ihre Energie in Form von Photonen abgeben 
und in einen Zustand der Hauptquantenzahl 2 zerfallen. Da in dem Gas zusätzlich Deuterium
vorzufinden ist, kann die Isotopieaufspaltung, die zur Hyperfeinstruktur zählt,
der Balmer-Serie untersucht werden.
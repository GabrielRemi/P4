\section{Fazit}\label{sec:fazit}
Im ersten Versuchsteil wurde mithilfe des Photoeffekts das Plancksche Wirkungsquantum
bestimmt. Dabei wurde eine Photozelle mit Licht unterschiedlicher Wellenlängen 
bestrahlt und mit der Gegenfeldmethode der dabei enstehende Photostrom gemessen.
Für jede Wellenlänge wurde die Grenzspannung aus der dazugehörigen 
Kennlinie bestimmt, wobei hier für hohe Wellenlängen die hierfür benötigte
Spannung deutlich kleiner war als zur Verfügung stand, weswegen hier
die Messung zu ungenau ausfiel und eine potentielle Fehlerquelle darstellt.
Durch Variation der angeschlossenen Widerstände würde es sich anbieten, 
hier die maximale Spannung zu minimieren um die Messung empflindlicher 
zu machen.\par
Die Messung wurde für Lichteinstrahlung mit \SI{365}{\nano\meter} bei erhöhter
Intensität wiederholt, um die Intensitätsabhängigkeit der
Kennlinie zu untersuchen. Dabei konnte die Quantenhypothese bestätigt werden, 
da die Steigung der Kennlinie eine Proportionalität zur Intensität
des Lichts aufweisen konnte, die Grenzspannung jedoch unverändert blieb. 
Dies zeigt, dass Lichtenergie gequantelt an die Elektronen abgegeben wird 
mit der Relation $E=\mathrm h\nu$.
Des Weiteren konnte die Austritsarbeit der Platin-Rhodium-Anode auf 
\[\mathrm W_{\mathrm A} = \SI{1.75\pm0.11}{\electronvolt}\]
bestimmt werden, was mit keinem Literaturwert verglichen werden konnte. 
Eine potentielle Fehlerquelle nach unten ist hierbei die Raumtemperatur, die 
durch die Wechselwirkung mit den Elektroden ihre Austritsarbeit minimiert.\par
Das in diesem Versuch bestimmte Wirkungsquantum ist 
\[\mathrm h = \SI{6.4(3)e-34}{\joule\second} ,\]
wodurch der Literaturwert $\mathrm h = \SI{6.626}{\joule\second}$ im $1\sigma$-Bereich der Messung liegt.
\\ \par
In dem Versuchsteil zur Balmer-Serie wurde zunächst die Gitterkonstante des verwendeten Reflexionsgitters
mithilfe der bekannten Spektrallinien der Hg-Dampflampe bestimmt. Durch eine Geraden-Anpassung
ergab sich
\begin{equation*}
    g = \SI{407(5)}{\nano \meter} .
\end{equation*} Jedoch wurden für diese Geraden-Anpassung die Messergebnisse für die drei roten
Spektrallinien der Hg-Spektrallampe nicht berücksichtigt, da diese aufgrund ihrer zu geringen Intensität nur sehr ungenau
vermessen werden konnten, was in der Auswertung zu unbrauchbaren Ergebnissen führte. Dies hätte eventuell durch
eine präzisere Justierung und Einstellung des Spaltes behoben werden können.\par
Mithilfe dieser experimentell bestimmten Gitterkonstante lässt sich sofort mit den angegebenen Abmessungen des Reflexionsgitters
dessen Auflösungsvermögen zu $A = \num{61400(800)}$ bestimmen, was einer gerade noch auflösbaren Wellenlängendifferenz von
\begin{equation*}
    \Delta \lambda = \SI{8,1(1)e-12}{\nano \meter}
\end{equation*} entspricht. Mit diesem Auflösungsvermögen wäre es theoretisch möglich, die später zu bestimmende Isotopieaufspaltung
der Balmer-Spektrallinien genau zu beobachten. Jedoch wurde bei der Versuchsdurchführung nicht das gesamte Gitter ausgeleuchtet,
was zu einem geringeren Auflösungsvermögen geführt hat. Damit war es dann z.B. nicht mehr möglich, die Isotopieaufspaltung
der $\mathrm{H_{\gamma}}$-Linie aufzulösen. Auch dieses Problem hätte durch eine präzisere Einstellung des Spaltes gelöst werden können.\par
Außerdem wurde die nun bestimmte Gitterkonstante verwendet, um die unbekannten Wellenlängen der $\mathrm{H_{\alpha}}$-, $\mathrm{H_{\beta}}$- und
$\mathrm{H_{\gamma}}$-Linien der Balmer-Serie zu bestimmen. Hierbei ergaben sich die Werte
\begin{equation*}
    \lambda_{\mathrm{H_{\alpha}}} = \SI{649(5)}{\nano \meter}, \quad \lambda_{\mathrm{H_{\beta}}} = \SI{482(6)}{\nano \meter}, \quad \lambda_{\mathrm{H_{\gamma}}} = \SI{428(6)}{\nano \meter},
\end{equation*} welche im Rahmen der Unsicherheiten gut mit den Literaturwerten übereinstimmen (siehe \cref{tab:wellenlangen_balmer}).
Ebenso wurden gute Ergebnisse bei der Bestimmung der Isotopieaufspaltung erzielt, was in \cref{tab:aufspaltung_balmer} einzusehen ist. Mit einer
präziseren Einstellung des Spaltes hätte gegebenenfalls noch die Isotopieaufspaltung der $\mathrm{H_{\gamma}}$-Linie aufgelöst werden könnnen.\par
Diese Balmer-Spektrallinien wurden auch mit einer CCD-Kamera und dem zugehörigen Computer-Programm aufgezeichnet. Damit konnte ebenfalls
die Isotopieaufspaltung berechnet werden (siehe \cref{tab:aufspaltung_balmer_2}), indem an die Messwerte eine Überlagerung
zweier Gauß-Kurven angepasst wurde, was aber zu wesentlich schlechteren Ergebnissen führte. Auch
dies ist wieder auf eine zu ungenaue Justierung des Spaltes und des gesamten Versuchsaufbaus zurückzuführen. Anhand der Gauß-Kurven konnte
eingesehen werden, dass die bei der Messung beobachtete Linienbreite der Spektrallinien hauptsächlich durch die Doppler-Verbreiterung
bestimmt wird, da diese im Vergleich zu der natürlichen Linienbreite wesentlich größer ist (3 Größenordnungen). Details hierzu sind in
\cref{tab:doppler} einzusehen. Aus den bestimmten Wellenlängen der Balmer-Serie wurde dann die Rydberg-Konstante und damit
das Plancksche Wirkungsquantum zu
\begin{equation*}
    h = \SI{6,600(4)e-34}{\joule \second}
\end{equation*} bestimmt. Dieser Wert weicht leicht von dem Literaturwert ($h = \SI{6,626e-34}{\joule \second}$) ab, da die
Unsicherheit sehr klein ist. Eventuell wurden die Messunsicherheiten zu klein abgeschätzt.
Zusammengefasst konnten bei diesem Versuchsteil gute Ergebnisse erzielt werden, jedoch hätte eine noch präzisere Justierung einen Zugang
zu besseren Ergebnissen verschafft.
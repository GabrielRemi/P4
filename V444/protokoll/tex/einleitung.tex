\section{Einleitung}\label{sec:einleitung}
\pagenumbering{arabic}
In diesem Versuch wird die Oberflächenplasmonen-Resonanz-Spektroskopie genutzt,
um die Anregung von Oberflächen-Plasmon-Polaritonen (OPPs) an dünnen Gold- und Silberfilmen
zu beobachten. Durch evaneszente elektromagnetische Wellen, die in ihrer Amplitude 
mit zunehmender Schichtdicke in einem Medium exponentiell abfallen, können an 
Schichtübergängen zwischen einem Dielektrikum und einem Metall OPPs
angeregt werden, die einen Mischzustand aus elektromagnetischen Wellen und Elektronendichtewellen 
darstellen und selbst Energie abstrahlen. 
Die OPPs werden bei diesem Versuch mithilfe der Kretschmann-Konfiguration angeregt
und mit einem Zwei-Prismen-Aufbau untersucht.\par
Im ersten Versuchsteil wird durch Einstrahlung mit einem monochromatischen
Laser auf das Prisma mit Goldschicht der Reflexionsgrad bei unterschiedlichen Einstrahlungswinkeln gemessen,
woraus die Schichtdicke der aufgetragenen Schicht bestimmt werden kann.\par 
Im zweiten Versuchsteil wird der Laser mit einer Weißlichtquelle und die Gold- mit 
einer Silberschicht ersetzt, um die Intensität 
des gesamten Lichtspektrums im sichtbaren Spektralbereich zu messen, sodass aus den hierbei 
entstehenden Intensitätsminima die Dispersionrelation der OPPs ermittelt werden kann.
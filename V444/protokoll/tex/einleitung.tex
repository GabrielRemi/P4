\section{Einleitung}\label{sec:einleitung}
\pagenumbering{arabic}
In diesem Versuch wird die Oberflächenplasmonen-Resonanz-Spektroskopie genutzt,
um die Anregung von Oberflächenplasmonpolaritonen (OPPs) an dünnen Gold- und Silberfilmen
zu beobachten. Durch evaneszente elektromagnetische Wellen, die in ihrer Amplitude 
mit zunehmender Schichtdicke in einem Medium exponentiell abfallen, können an 
Schichtübergängen zwischen einem Dielektrikum und einem Metall Oberflächenplasmonen
angeregt werden, die einen Mischzustand aus elektromagnetischen und Elektronendichtewellen 
darstellen und selbst Energie abstrahlen können. 
Die OPPs werden bei diesem Versuch mithilfe der Kretschmann-Konfiguration angeregt
und mit einem zwei-Prismen Aufbau gemessen.\par
Im ersten Versuchsteil wird durch Einstrahlung mit einem 
Laser der Reflexionsgrad bei unterschiedlichen Einstrahlungswinkeln
bei Einstrahlung auf das Prisma mit Goldschicht gemessen, woraus 
die Schichtdicke der aufgetragenen Schicht bestimmt werden kann.\par 
Im zweiten Versuchsteil wird der Laser mit einer Weißlichtquelle und die Gold- mit 
einer Silberschicht ersetzt, um die Intensität 
des gesamten Lichtspektrums im sehbaren Bereich gemessen, um aus den hierbei 
enstehenden Intensitätsminima die Dispersionrelation der Plasmonen zu ermitteln.

\section{Fazit}\label{sec:fazit}
In diesem Versuch wurde die Oberflächenplasmonen-Resonanz-Spektroskopie genutzt, um die Dicken von insgesamt vier Gold-Filmen und
die OPP-Dispersionsrelation von Silber zu bestimmen. Dafür wurde die Kretschmann-Konfiguration verwendet, bei der die zu untersuchende
Probe auf der Hypotenuse eines rechtwinkligen Prismas befestigt wird und das einfallende Licht an der Prisma-Metall-Grenzfläche reflektiert
und anschließend untersucht wird.\par
Im ersten Versuchsteil wurde ein monochromatischer Diodenlaser ($\lambda = \SI{785}{\nm}$) verwendet, um über den gemessenen Reflexionsgrad
die Dicken von insgesamt vier Gold-Filmen zu ermitteln. Dafür war zunächst eine Referenzmessung notwenig, über die der Brechungsindex
des Prismas bestimmt werden sollte. Bei der Versuchsdurchführung konnten keine sinnvolle Messergebnisse (für die reflektierten Intensitäten der
Goldfilme) erreicht werden, da die verwendeten Gold-Filme vermutlich schon zu veraltet/verschmutzt waren. Aus diesem Grund wurden vom Assistenten
Messwerte (auch für die Referenzmessung) zur Verfügung gestellt, um dennoch eine sinnvolle Auswertung zu erarbeiten. So konnten aus der Referenzmessung
die Brechungsindizes
\begin{equation*}
    n_{\mathrm{Prisma,TM}} = \num{1,50028(12)}
\end{equation*} für TM-Polarisation und
\begin{equation*}
    n_{\mathrm{Prisma,TE}} = \num{1,50047(13)}
\end{equation*} für TE-Polarisation bestimmt werden. Diese Ergebnisse sind sehr plausibel, da sie konsistent untereinander sind und einem typischen Glas-Brechungsindex
von \num{1,5} sehr nahe kommen. Bei den Gold-Filmen konnten über den gemessenen Reflexionsgrad und eine geeignete Anpassung die zugehörigen Dicken bestimmt werden,
welche in \cref{tab:gold_fazit} dargestellt sind.
\begin{table}[H]
    \centering
    \caption{Experimentell bestimmte Dicken der vier Gold-Filme.}
    \begin{tabular}{c|c}
        Goldfilm & $d_{\mathrm{Gold}}$ / \unit{\nm} \\
        \hline
        1 & $\num{29,80(14)}$ \\
        2 & $\num{50,48(19)}$ \\
        3 & $\num{75,2(4)}$ \\
        4 & $\num{42,50(16)}$               
    \end{tabular}\label{tab:gold_fazit}
\end{table} Auch hier wurden sehr plausible Ergebnisse erzielt, da die Dicke eines Gold-Films in dieser Größenordnung liegen muss, sodass OPPs angeregt werden können.
Außerdem konnte insgesamt festgestellt werden, dass nur für TM-Polarisation OPPs angeregt werden.
Zusammenfassend hätte dieser Versuchsteil in der Durchführung besser funktioniert, hätten neuere Goldproben zur Verfügung gestanden. Dafür konnten
die zur Verfügung gestellten Messwerte erfolgreich ausgewertet werden.\par
Im zweiten Versuchsteil wurde der monochromatische Diodenlaser durch eine Halogenlampe (Weißlichquelle) ersetzt, um die OPP-Dispersionsrelation eines
Silber-Films zu bestimmen, der gemeinsam mit dem Assistenten vor der Versuchsdurchführung aufgedampft wurde. Auch hier hat sich zunächst gezeigt, dass
nur für TM-Polarisation OPPs angeregt werden können. Über die aufgenommenen Reflexionsspektren für TM-Polarisation konnte über die daraus bestimmbaren
OPP-Anregungswellenlängen die OPP-Dispersiosnrelation für Silber bestimmt werden. Ein Vergleich mit der Lichtlinie (siehe \cref{fig:darstellung_dispersionsrelation})
zeigt hier, dass die experimentell bestimmten Ergebnisse den theoretischen Erwartungen entsprechen, da sich die OPP-Dispersionsrelation (im sichtbaren Spektralbereich)
nicht mit der Lichtlinie schneiden sollte. So wurde also experimentell für den sichtbaren Spektralbereich bestätigt, dass die Kretschmann-Konfiguration
notwendig ist, um OPPs anzuregen, da dann die Anregungs-Bedingung $k_{\mathrm{SPP,||}} = \frac{\omega}{c_0} n_{\mathrm{Prisma}}$ erfüllt werden kann.